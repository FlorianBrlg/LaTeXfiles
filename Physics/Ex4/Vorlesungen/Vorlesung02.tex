\section{Lorentz Modell}

fig 1.1\\
Dipolmoment: $d = -ex(t)$\\
Harmonischer Oszillator: $x(t) = x_0 \cos(\omega t)$

\subsubsection{Abgestrahlte Leistung}
$$
P = \frac{e^2 x_0^2 \omega^4}{12\pi\epsilon_0 c^3}
$$
Dies vergleichen wir mit der Energie des Elektrons selber
$$
E = \frac{m}{2}\omega^2 x_0^2
$$
und somit
$$
\frac{\text dE}{\text dt} = -P = \underbrace{-\frac{e^2\omega^2}{6\pi\epsilon_0 mc^3}}_{1/\tau}\cdot E = -\frac{E}{\tau}
$$
also fällt die Energie exponentiell ab

\subsubsection{Natrium}
D-Linie: $\lambda = 589$nm\\
$\Gamma = \frac{1}{\tau} = 6\cdot 10^{7}$ und der experimentelle fall\\
$\Gamma_{exp} = 6\cdot 10^{7}$
\subsubsection{Wasserstoff}
$\lambda = 121$nm
\begin{gather*}
	\Gamma = 1.5\cdot10^{9}\\
	\Gamma_{exp} = 6\cdot10^{8}
\end{gather*}
und für Yb $\lambda = 467$nm
\begin{gather*}
	\Gamma = 10^{8}\\
	\Gamma_{exp} = 10^{-7}
\end{gather*}

\subsection{Einstein Ratengleichungen}

Seien zwei zustände $E_2, E_1$ mit $E_1$ Grundzustand.\\
fig 1.2\\

Absorption: die änderung von $E_2$
$$
\frac{\text dN_2}{\text d t} = N_1\cdot B_{12} \cdot \rho(\omega)
$$
für Stimulierte Emission
$$
\frac{\text dN_2}{\text dt} = -N_2\cdot B_{21}\cdot\rho(\omega)
$$
für Spontane Emission
$$
\frac{\text dN_2}{\text dt} = -N_2\cdot A_{21}
$$
und wir wissen über $N_1$ dass
$$
\frac{\text dN_1}{\text dt} = -\frac{\text dN_2}{\text dt}
$$

fig 1.3

Das Plank'sche Strahlungsgesetz
$$
\rho(\omega) = \frac{\hbar \omega^3}{\pi^2 c^3} \frac{1}{\exp\left(\frac{\hbar\omega}{k_BT}\right)-1}
$$
und wir finden dass
\begin{gather}
	A_{21} = \frac{\hbar\omega_{12}^3}{\pi^2 c^3} B_{21}
\end{gather}
mit $B_{21}$ als Intrinsische Eigenschaft des Atoms von Einstein her.

\subsection{Atome in einem Magnetfeld}

Das bewegte Elektron ist ein Kreisstrom.
$$
I = -\frac{er}{2\pi r}
$$
besitzt ein Magnetisches moment
$$
\mu = -\frac{1}{2}e\vec r\times \vec v
$$

Ein Magnetisches Moment in einem externen Magnetfeld $B$ hat die Wechselwirkungsenergie (Potentielle Energie)
$$
V = -\vec\mu \cdot\vec B
$$

\subsection{Lorentz Modell}

$$
m\frac{\text dr}{\text dt} = \underbrace{-m\omega_0^2 r}_{H.O.} - \underbrace{er\times B}_{\text{Lorentzkraft}}
$$
wir definieren $B = B_0 e_z$
\begin{equation}
	\ddot r + 2\Omega_L \dot r \times e_z + \omega_0^2 r = 0\;\;\;\;\; \Omega_L = \frac{eB}{2m}
\end{equation}
Wir lösen mit dem Ansatz
\begin{equation}
	r(t) = (xe_x + ye_y + ze_z) e^{-i\omega t}
\end{equation}
und die matrixform ist dann
\begin{equation}
	\begin{pmatrix}
		\omega_0^2 & -2i\omega\Omega_L & 0\\
		2i\omega\Omega_L & \omega_o^2 & 0\\
		0 & 0 & \omega_0^2
	\end{pmatrix}
	\begin{pmatrix}
		x \\ y \\ z
	\end{pmatrix}
	= \omega^2
	\begin{pmatrix}
		x \\ y\\ z
	\end{pmatrix}
\end{equation}
Eigenwertgleichung mit char. Polynom
\begin{equation}
	\left[\omega^4 - (2\omega_0^2 + 4\Omega_L^2)\omega^2 + \omega_0^4\right](\omega^2 - \omega_0^2) = 0
\end{equation}
Wir nehmen an, dass $\Omega_L << \omega_0$. So finden wir dass $\omega \approx \omega_0 \pm \Omega_L$, und wir erhalten die Eigenvektoren
\begin{equation}
	\begin{matrix}
		\omega = \omega_0 - \Omega_L & \omega=\omega_0 & \omega = \omega_0+\Omega_L\\
		\begin{pmatrix}
			\cos(\omega t)\\
			-\sin(\omega t)\\
			0
		\end{pmatrix}		

		& z_0
		\begin{pmatrix}
			0 \\ 0 \\ \cos(\omega_0 t)
		\end{pmatrix}
		&
		\begin{pmatrix}
			\cos(\omega t)\\
			\sin (\omega t)\\
			0
		\end{pmatrix}
	\end{matrix}
\end{equation}

\chapter{Quantenmechanik}
\begin{itemize}
	\item diskrete Energieniveaus
	\item Abstrahlung von Energie führt zu Änderung des Zustands des Elektrons 
\end{itemize}
\subsection*{Offene Fragen}
\begin{itemize}
	\item Auspaltung der D-Linie in Natrium.
	\item Anormaler Zeemaneffekt (Aufspaltung in Gerade Anzahl von Linien)
\end{itemize}

\subsection{Doppelspaltexperiment}

Konstruktive Interferenz für
$$
\Delta \ell = n\lambda, \;\;\; n = 0, \pm1, \pm2,\cdots
$$
desuktrive Interferenzen für
$$
\Delta \ell = (n+1/2)\lambda
$$
Die Intensität auf dem Schirm ist proportional zum Betragsquadrat der
\begin{gather}
	I \propto |E_1 + E_2|^2\\
	I \propto |E_1|^2 + |E_2|^2 + \underbrace{2\text{Re}(E_1* E_2)}_{\text{Interferenz}}
\end{gather}

Was passiert mit einem Teilchen auf dem Doppelspalt?\\
deBroglie: Auch Teilchen haben Welleneigenschaften
$$
\lambda = h/p
$$
Einstein: Energie Impuls Beziehung
$$
E = \hbar\omega = \sqrt{(mc^2)^2 + p^2c^2}
$$
und für Masselose teilchen $E=pc = \hbar kc = \hbar\omega$ und somit ist $\omega = ck$\\
Für massive Teilchen $E = mc^2 \sqrt{1 + \frac{p^2}{m^2c^2}} \approx mc^2 + \frac{p^2}{2m}$ und damit gilt dann $\omega = \frac{\hbar k^2}{2m}$ mit kinetischer Energie kleiner als $mc^2$.

