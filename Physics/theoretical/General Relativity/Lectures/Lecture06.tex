\chapter{Principle of Equivalence}

\section{Special Relativity Pros and Cons}

Pros
\begin{itemize}
	\item composition of velocity
	\item $E=\gamma mc^2$
	\item time dilation
	\item unify with QM for QFT
\end{itemize}
Cons
\begin{itemize}
	\item notion of absolute spacetime
	\item What happens in accelerated frames?
\end{itemize}

\section{Principle of Equivalence}
inertial mass: Resistance against acceleration
\begin{equation}
	F = m_i a  = m_i \diff{v}{t}, \;\; m_i: \text{Intertial Mass}
\end{equation}
Gravitational mass: factor of proportionality in gravitational force

\subsection{Galilei (Newton, Huggels)}
forces fall at same rate independent of their mass. In a free Fall:
\begin{equation}
	m_i a = m_g g \rightarrow a = \frac{m_g}{m_i}g
\end{equation}
Observation: $\frac{m_g}{m_i}$ is universal, $g$ can be defined such that $m_i = m_g$

\subsection{Eötvös Experiment (1889)}
(Weinberg)
Simplified picture:\\
revolving heavy disk with a scale that balances vertical forces and at the same time radial forces.\\
The torque pendulum: There are two masses on a scale held up by a string that can freely rotate. Let $F_g$ denote the respective gravity force on the mass, and let $F_C$ be the centrifugal forces on the masses. The gravitational balance is
\begin{equation}
	l_A m_{gA} = l_B m_{g_B}
\end{equation}
The resulting torque through centripetal force
\begin{equation}
	T = (l_A m_{iA} - l_B m_{iB})g_i
\end{equation}
Then $g_i$ is thhe acceleration due to the centripetal force. Observation: $T=0$, so we take that $m_i = m_g$

\subsection{Einstein Thought Experiment}
freely falling elevator of mass $m$.\\
The external observer sees 
$$
m\diff{^2x}{t^2} = mg
$$
The internal observer sees
$$
x' = x - \frac{1}{2}gt^2
$$
Then we compute $x'$ and we find
$$
m\diff{^2 x'}{t^2} = 0
$$
The principle of equivalence tells us that the gravitational and inertial forces can be cancelled for all freely falling objects.

The strong equivalence principle (SEP)(Popular in exam): At every spacetime point in an arbitrary gravitational field it is possible to choose a locally inertial coordinate system such that, within a sufficiently small region around the point in question the laws of nature take the same form as in an unaccelerated cartesian coordinate system in the absence of gravity 

Effects of gravitational field can be detected by non-local experiment. Applied to measure earth gravitational field GOCE. Gravitational wave detection is non local.

The weak equivalence principle: Laws of nature replaced by laws of motion ($m_i = m_g$).

Einstein Equivalence Principle: Laws of nature replaced by non gravitational laws of nature

Gravitational binding energy affects $m_g$ differently from $m_i$

\subsection{gravitational red shift (from SEP or EEP)}
Let $1$ and $2$ be spaceships with a constant distance $z$, while accelerating with $a$. $1$ sends a photon to $2$, time between being sent and being detected is $\Delta t = z/c$. Gained speed during $\Delta t$ is $\Delta v = a\Delta t = \frac{az}{2}$. Incremental Doppler Shift $\frac{\Delta v}{c} = \frac{\Delta \lambda}{\lambda_0} = \frac{az}{c}$
\subsubsection{Equivalent setup with a=g}
let $1$ and $2$ be observers, where $2$ is on top of a tower. $1$ sends a photon to $2$. Applying the previous relation we see that
$$
\frac{\Delta\lambda}{\lambda_0}= \frac{gz}{c^2}
$$
is the gravitational redshift.\\
Intuition: The energy of a photon is given by $h\nu$, the change is $\Delta(h\nu) = -\frac{gz}{c^2}h\nu$. The change of energy in a photon is the change of potential energy $(mgz = \frac{h\nu}{c^2}hz$ if $m\rightarrow \frac{E}{c^2}$.

\subsubsection{Observation: Pound \& Rebka (1960)}
let $\gamma$ ray from $Fe^{57}$ fall 22.6 meters. Resonant absorption showed
\begin{gather}
	\frac{\Delta \nu}{\nu} = (2.57 \pm 0.26)\cdot 10^{-15}\\
	\text{With the theory: } \frac{\Delta \nu}{\nu} = 2.46\cdot 10^{-15}
\end{gather}


\chapter{Manifolds}
\section{Definitions and Examples}
Topologies generalize the notion of open intervalls, balls, etc

\subsection{Def:}
Let there be a collection of subsects $\tau = \left\{U_i|i\in I\right\}$ in a set $X$ such that it possesses the following properties:
\begin{itemize}
	\item $\emptyset, X \in \tau$
	\item for any $J\subset I$ we have that $\cup_J U_j \in \tau$.
	\item for any finite $J\subset I$ we have that $\cap_J U_j \in \tau$.
\end{itemize}
Then $I,J$ are index sets, $\tau$ is a topology in $X$, and $X$ is a topological space. The elements of $\tau$ are called open subsets. For example
\begin{itemize}
	\item trivial topology $\tau = \{\emptyset, X\}$
	\item discrete topology $\tau = \{\text{all subsets in }X\}$
	\item Euclidean topology $X=\mathbb R^n, I=\mathbb R^n$, $U_y$ is the open ball with radius $r$ and all unions and intersections
\end{itemize}














