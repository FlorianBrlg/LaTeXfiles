\section{last time}

\begin{gather}
	\Lambda_\mu^\rho \Lambda_\nu^\kappa \eta_{\rho\kappa} = \eta_{\mu\nu}\\
	x^\mu\mapsto \Lambda^\mu_\nu x^\nu\\
	a_\mu = \eta_{\nu\mu}a^\nu\\\
	(\text d\tau)^2 = -(\text ds)^2
\end{gather}

\subsection{Examples}

4-vector: $x^\mu = (ct, \vec x)$.\\
4-velocity: $\frac{\text dx^\mu}{\text d\tau} = u^\mu$
\begin{gather}
	\vec x = \vec v t,\; u^\mu = (\frac{\text dt}{\text d\tau}), \vec u = \frac{\text dt}{\text d\tau})\\
	\text d\tau^2 = \text dt^2 - \text dx^2,\;\; \text d\tau = \frac{1}{\gamma}\text dt\\
	u^\mu = \gamma(1,\vec v),\;\;\; u_\mu u^\mu = -1
\end{gather}
4-acceleration: $a^\mu = \frac{\text d^2 x^\mu}{\text d\tau^2}, \;\;\; a_\mu u^\mu = 0$\\

\subsubsection{Tensors}
\begin{equation}
	\tilde T^{\mu_1,\cdots,\mu_k}_{\nu_1,\cdots,\nu_m} = \Lambda^{\mu_1}_{\kappa_1}\cdots\Lambda^{\mu_k}_{\kappa_k}\Lambda^{\lambda_1}_{\nu_1}\cdots\Lambda^{\lambda_m}_{\nu_m}T^{\kappa_1,\cdots,\kappa_k}_{\lambda_1,\cdots,\lambda_m}
\end{equation}
\subsubsection{Examples for Tensors}
Electrodynamics $E$, $B$ not part of $4$ vectors.
\begin{equation}
	F_{\mu\nu} \partial_\mu A_\nu - \partial_\nu A_\mu, \;\; B_i = \epsilon_{ijk}F_{jk}, \;\; E_i = F_{i0}
\end{equation}
$\eta_{\mu\nu}$ invariant.\\
$\epsilon_{\mu\nu\rho\lambda}$ the totally anti-symmetric pseudo-tensor, invariant.
\begin{gather}
	\tilde \epsilon_{\mu\nu\rho\lambda}=\det(\Lambda) \underbrace{\Lambda_\mu^\kappa \Lambda_\nu^\phi \Lambda_\rho^\psi \Lambda_\lambda^\xi \epsilon_{\kappa\phi\psi\xi}}_{\text{totally anti-symmetric}}\\
	= \det(\Lambda)\epsilon_{\mu\nu\rho\lambda}\Lambda_0^\kappa \Lambda_1^\phi \Lambda_2^\psi \Lambda_3^\xi \epsilon_{\kappa\phi\psi\xi} = \det(\Lambda)^2 \epsilon_{\mu\nu\rho\lambda}
\end{gather}
the determinant is the cause for the pseudo-tensor property.


\chapter{Relativistic Field Theory}
-Relativistic Quantum Field Theory $\rightarrow$ Particle Physics\\
-Here, Classical Field Theories, only two relevant examples. That is, Electrodynamics and General Relativity.

\section{Euler Lagrange Equation}
For simplicity, one scalar field $\tilde\phi(\tilde x^\mu) = \phi(x^\mu)$.
\begin{gather}
	S = \int \text dt L = \int \text d^4 x \mathcal L \left(\phi(x^\mu),\partial_\mu\phi(x^\mu)\right)
\end{gather}
(The derivative is also a 4-vector with a lower index.)\\
Least Action Principle; the action should be extremal, so the variation should be 0.
\begin{gather}
	0 = \delta S = \int \text d^4x \frac{\partial\mathcal L}{\partial\phi}\delta \phi + \frac{\partial\mathcal L}{\partial \partial_\mu\phi}\delta\partial_\mu\phi
\end{gather}
Now integrate by parts and choose boundary conditions so that the contributions vanish.
\begin{gather}
	0 = \int \text d^4 x \left(\frac{\partial\mathcal L}{\partial \phi} - \partial_\mu \frac{\partial\mathcal L}{\partial \partial_\mu\phi}\right)\delta\phi\\
	\Rightarrow \partial_\mu\frac{\partial\mathcal L}{\partial\partial_\mu\phi} - \frac{\partial\mathcal L}{\partial\phi} = 0
\end{gather}

\subsection{Examples}
\begin{itemize}
	\item Massive real scalar field (e.g. Higgs)
		$$
		\mathcal L = \frac{-1}{2}\partial_\mu \phi \partial^\mu \phi - \frac{m^2}{2}\phi^2
		$$
	\item Electromagnetic field
		$$
			S = \int\text d^4x\left(-\frac{1}{4}F_{\kappa\lambda}F^{\kappa\lambda} + j_\lambda A^\lambda\right)
		$$
\end{itemize}

\section{Noether Theorem}
For every continuous global symmetry there exists one conserved current.\\
\textit{Proof: } A continuous symmetry is parametrized by a continuous parameter (e.g. rotation angle). Global means that the symmetry does not depend on space or time.\\
Consider one parameter symmetry, call parameter $a$. Consider$a(x^\mu)$ non-constant, then
$$
\delta_a S \int \text d^4x (\partial_\mu a)j^\mu.
$$
Integration by parts, $a_{\partial r}=0$.
$$
\delta_a S = -\int \text d^4x a\partial_\mu j^\mu.
$$
If the Equations of Motions are satisfied then $\delta S = 0$ for any $a$.
$$
0 = \int \text d^4x a\partial_\mu j^\mu
$$
for any $a \rightarrow \partial_\mu j^\mu = 0$.

\subsection{Noether Charge}
\begin{gather}
	Q = \int \text d^3x j^0\\
	\frac{\text dQ}{\text dt} = \int \text d^3x \partial_0\partial^0 = \text d^3x \div j = 0
\end{gather}

\subsection{examples}
Complex Scalar Field
\begin{equation}
	\mathcal L = -\frac{1}{2}\partial_\mu \phi \partial^\mu \phi^\dagger - m^2 \phi\phi^\dagger
\end{equation}
invariant under complex phase change.
\begin{gather}
	e^{i\alpha}\phi \approx (1+i\alpha)\phi\\
	e^{-i\alpha}\phi \approx (1-i\alpha)\phi\\
	\delta_\alpha S = -\partial_\mu(i\alpha\phi)\partial^\mu\phi^\dagger - \partial_\mu\phi\partial^\mu(-i\alpha\phi^\dagger) - m^2i\alpha\phi\phi^\dagger - m^2\phi(-i\alpha\phi^\dagger)\\
	= \partial_\mu\alpha\left(-i\phi\partial^\mu\phi^\dagger + i(\partial^\mu\phi) \phi^\dagger\right)
\end{gather}
So then we can see that
\begin{equation}
	j^\mu = -i\phi\partial^\mu\phi^\dagger + i(\partial_\mu \phi)\phi^\dagger
\end{equation}
This would be the electromagnetic current.

\subsubsection{Remark}
if the action is invariant under a local symmetry then the Noether Theorem does not apply. However, conserved currents can also come from the Equations of motion.\\
e.g.
\begin{equation}
	\partial_\mu F^{\mu\nu} = j^\nu,\;\;\; \partial_\nu j^\nu = \partial_\nu\partial_\mu F^{\mu\nu} = 0,\;\;\;\text{Noether's 2nd Theorem}
\end{equation}
conservation laws appears as integrability condition on equation of motion 
\begin{equation}
	A_\mu\rightarrow A_\mu + \partial_\mu \alpha
\end{equation}

\subsection{Energy Momentum}
symmetry $x\mu \rightarrow x^\mu + a^\mu








