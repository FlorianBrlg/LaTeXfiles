
- basis in $(k,l)$ tensor space = tensor product of $(1,0)$ and
$$
T = T^{\mu_1\cdots\mu_k}_{\nu_1\cdots\nu_l} \partial_{\mu_1}\otimes\cdots\otimes\partial_{\mu_k}\otimes\text dx^{\nu_1}\otimes\cdots\otimes\text dx^{\nu_l}
$$
example $(2,0)$ tensor
$$
T(\omega^1, \omega^2) = T^{\mu_1\mu_2}\partial_{\mu_1}\otimes \partial_{\mu_2}\left(\omega_{\nu_1}^1 \text dx^{\nu_1}, \omega^2_{\nu_2}\text dx^{\nu_2}\right)
$$

\section{Manipulating Tensors}

\subsection{Tensor product}

\begin{gather}
	T\otimes S(\omega^1, \cdots, \omega^{k+m},V^{1}, \cdots, V^{l+n}) =\\
	T(\omega^1,\cdots,\omega^k,V^1,\cdots, V^l) S(\omega^{k+1}, \cdots,\omega^{k+l},V^{l+1},V^{l+n})
\end{gather}
in components
$$
\left(T\otimes S\right)^{\mu_1\cdots\mu_{k+m}}_{\nu_1\cdots\nu_{l+n}} = T^{\mu_1\cdots\mu_k}_{\nu_1\cdots\nu_l}S^{\mu_{k+1}\cdots\mu_{k+m}}_{\nu_{l+1}\cdots\nu_{l+n}}
$$
\subsection{contraction}
first for $(1,1)$ tensor. $T:T_p* \times T_p\rightarrow \mathbb R$\\
we can view $T\in \text{Span}(T_p\otimes T_p*)$\\
and the contraction $C: T_p\otimes T_p*\rightarrow \mathbb R$. $C(V,\omega) = \omega(V).$
$$
T = T^\mu_\nu \partial_\mu \otimes \text dx^\nu
$$
and
$$
C(T) = T^\mu_\nu C(\partial_\mu,\text dx^\nu) = T^\mu_\nu \text dx^\nu(\partial_\mu) = T^\mu_\mu
$$
Generalization to $T\in\text{Span}(T_p\otimes\cdots\otimes T_p\otimes T_p*\otimes\cdots\otimes T_p*)$ with $k$ vector spaces and $l$ covector spaces. Pick a pair $T_p, T_p*$ and treat that as in the $(1,1)$ case $\Rightarrow$ $(k-1,l-1)$ tensor.






