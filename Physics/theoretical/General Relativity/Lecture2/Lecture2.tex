\documentclass[]{scrartcl}

\usepackage{\string~"/LaTeX/StylePackage"}

\title{Lecture 2 - General Relativity}
\author{}
\date{9.4.2025}


\begin{document}

\maketitle
\newpage
\tableofcontents
\newpage

\subsection{twin Paradox} 
proper time of
\begin{gather}
	\text{twin 1: } \Delta \tau_{ABC} = \Delta t\\
	\text{twin 2: } \Delta\tau_{AB'C} = \Delta\tau_{AB'} + \Delta\tau_{B'C}\\
	\Delta\tau_{AB'} = \sqrt{1-v^2} \frac{\Delta t}{2} = \Delta\tau_{B'C}
\end{gather}
Therefore
\begin{equation}
	\Delta\tau_{AB'C} = \sqrt{1-v^2}\Delta\tau_{ABC}
\end{equation}
travelling twin is younger


\section{Lorentz Transformations}

\begin{gather}
	\tilde x^\mu = \Lambda^\mu_\nu x^\nu
\end{gather}
impose that the Minkowski Distance from the origin is invariant, such that
\begin{equation}
	\tilde x^\mu \tilde x^\nu \eta_{\mu\nu} = x^\mu x^\nu \eta_{\mu\nu}
\end{equation}
Then
\begin{equation}
	\eta_{\mu\nu} \Lambda^\mu_\kappa \Lambda^\nu_\lambda = \eta_{\kappa\lambda}
\end{equation}

This property describes the Lorentz Group $0(3,1)$. In this Group there exist 4 branches.
\begin{gather}
	\kappa=\lambda=0\;\;\Rightarrow\;\; \eta_{00} = -1 = -(\Lambda^0_0)^2 + \underbrace{(\Lambda^i_0)^2}_{\geq 0}\\
	\Rightarrow\;\; \Lambda^0_0 \geq 0 \text{ or } \Lambda^0_0 \leq -1
	\text{from Matrix equation } \Lambda^T \eta \Lambda = \eta\\
	\Rightarrow \;\; \det(\Lambda^T \eta\Lambda) = \det(\eta) \;\;\Rightarrow\;\; \det(\Lambda)^2 =1
\end{gather}
then we get the four branches
\begin{itemize}
	\item proper Lorentz group (contains the identity) $\Lambda^0_0,\det\Lambda = 1$
	\item combine with time reversal $\Lambda^0_0 \leq -1,\det\Lambda = -1$
	\item combine with parity (e.g. $x\rightarrow -x$) $\Lambda^0_0 \geq 1,\det\Lambda = -1$
	\item time reversal and parity $\Lambda^0_0 \leq -1,\det\Lambda = +1$
\end{itemize}
Raising and lowering an index with the (Minkowski) metric tensor
\begin{equation}
	a_\mu = \eta_{\mu\nu} a^\nu
\end{equation}
and we have that $\eta_{\mu\nu} = \eta^{\mu\nu}$



\end{document}

