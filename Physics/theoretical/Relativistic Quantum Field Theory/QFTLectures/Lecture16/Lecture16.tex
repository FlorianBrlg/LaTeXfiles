%!TeX program = lualatex
\documentclass[]{scrartcl}

\usepackage{\string~"/LaTeX/StylePackage"}

\title{QFT - Lecture 16}
\author{}
\date{16.10.2024}


\begin{document}

\maketitle
\newpage
\tableofcontents
\newpage

\section{The scattering (S) matrix} 

Interaction can involve:
\begin{itemize}
	\item self-interaction
	\item scattering
	\item production of new particles
\end{itemize}

\section{Eigenstates of interacting theories $(\phi^4)$}
Hamiltonian $H$.\\
Momentum Operator $P$.\\
From the poincaré symmetry: $[H,P] = 0$.\\
Assumptions:
\begin{itemize}
	\item there exists a unique, translationally invariant and Lorentz invariant state $\ket\Omega$. The Energy zero is chose such that $H\ket\Omega = 0$
	\item $\bra\Omega \phi(x) \ket\Omega = 0$.
\end{itemize}

The common Eigenstates of $P$ and $H$, $\ket{\lambda_p}$ with $p$ the total momentum and $\lambda$ the degrees of freedom of the state.
\begin{equation}
	H\ket{\lambda_p} = E_p^\lambda\ket{\lambda_p},\;\;\;\; P\ket{\lambda_p} = p\ket{\lambda_p}
\end{equation} 
let $m_\lambda := E_p^\lambda I_{p=0}$ be the rest Energy.\\
Consider a Lorentz transformation from $(m_\lambda, 0) \mapsto (p^0, p)$\\
$$
	U^\dagger P^\mu U = \Lambda^\mu_\nu P^\nu
$$
\begin{gather}
	P^\mu U\ket{\lambda_0} = UU^\dagger P^\mu U \ket{\lambda_0}\\ 
	= U\Lambda^\mu_\nu P^\mu \ket{\lambda_0} = U \Lambda^\mu_0 m_\lambda\ket{\lambda_0}\\
	= p^\mu U\ket{\lambda_0}
\end{gather}
And therefore we conclude that $\ket{\lambda_p} = U\ket{\lambda_0}$

As the Lorentz transformation leaves Four-Vector length invariant, we have\\
\begin{gather}
	 (p^0)^2 - p^2 = m_\lambda^2\\
	 E_p^{\lambda^2} = m_\lambda^2 + p^2
	 m = m_\lambda\text{ for a single particle}
\end{gather}
A single particle starts at the Energy $m$, and is only on the shell. This is the mass gap. A second particle is at the Energy $2m$, and through internal motion can have any Energy above the shell.
\begin{gather}
	\phi(x) = e^{iPx}\phi(0) e^{-iPx}\\
	\bra\Omega \phi(x) \ket{\lambda_p} = \bra\Omega e^{iPx}\phi(0)e^{-iPx}\ket{\lambda_p}\\
	= \bra\Omega \phi(0) \ket{\lambda_p}e^{-ipx}\\
	= \bra\Omega UU^\dagger \phi(0) U \ket{\lambda_0}e^{-ipx}\\
	= \bra\Omega \phi(0) \ket{\lambda_0} e^{-ipx}\\
	\text{when $\lambda$ is a single particle, $\ket\lambda_0 = \ket{p=0}$.}\nonumber\\
	Z := \bra\Omega\phi(0)\ket{p=0}^2
\end{gather}

\section{Defining in and out states}
\begin{gather}
	\ket{\lambda_p^\psi} = \int \frac{\text d^3k}{(2\pi)^3}\psi(k-p)\ket{\lambda_k}\\
	\ket{p^\psi} = \int \frac{\text d^3k}{(2\pi)^3}\psi(k-p) \ket k
\end{gather}
Now Define Operators: $a_p^\psi(t) = i\int \bar\Psi_p(x) \overleftrightarrow\partial_0 \phi(x)$\\
$a_p^{\psi\dagger}(t) = -i\int \Psi(x) \overleftrightarrow\partial_0 \phi(x)$

Where we have define $\Psi(x) = \int \frac{\text d^3k}{(2\pi)^3}\frac{\psi(k-p)}{\sqrt{2E_k}}e^{-ikx}$ with $k^0$ = $E_k = \sqrt{k^2 + m^2}$.\\
Limits:
\begin{itemize}
	\item $t\rightarrow\pm\infty$
	\item $\psi(k) \rightarrow (2\pi)^3\delta(k)$
\end{itemize}
Result:
\begin{gather}
a_p^\psi(\pm\infty), a_p^{\psi\dagger}(\pm\infty)	
\end{gather}
Work as ladder operators for vacuum and single particles. (First two steps of the ladder)
\begin{gather}
	\ket{p} = \frac{\sqrt{2E_p}}{\sqrt{Z}}a_p^{\psi\dagger}(\pm\infty)\ket\Omega\\
	\braket{p|q} = 2E_p (2\pi)^3\delta(p-q)\\
	a_p^\psi(\pm\infty)\ket\Omega = 0
\end{gather}
Asymptotic in and out states:
\begin{gather}
	\ket{k_1,k_2,\cdots}_{\text{in}} = \prod_i \sqrt{\frac{2E_{k_i}}{Z}}a_{k_i}^{\psi\dagger}(-\infty)\ket\Omega\\
	\ket{p_1,p_2,\cdots}_{\text{out}} = \prod_j \sqrt\frac{2E_{p_j}}{Z} a_{p_j}^{\psi\dagger}(+\infty)\ket\Omega
\end{gather}
Both of these states are in the Heisenberg picture.

\section{Scattering in Schrödinger Picture}
let the reference time be $t=-\infty$. Consider Evolution from $t=-T$ to $t=T$: $S = e^{-iH2T}$.\\
For $t = -T$ in Schrödinger picture: $\ket{k_1,k_2,\cdots} = \ket{k_1,k_2,\cdots}_\text{in}$ from the Heisenberg picture.\\
For $t = T$ in Schrödinger picture: $\ket{p_1, p_2, \cdots} = S\ket{p_1,p_2,\cdots}_\text{out}$.\\
We define the S-Matrix as the overlap of the in and out states
\begin{gather}
S_m = \braket{p_1,p_2,\cdots|_\text{out}k_1,k_2,\cdots}_\text{in} = \braket{p_1,p_2,\cdots|_\text{out}S^\dagger S | k_1,k_2,\cdots}_\text{in}\\
= \braket{p_1,p_2,\cdots | S|k_1,k_1,\cdots}
\end{gather}

\subsubsection{Errors in Textbooks about the LSZ Reduction}
\begin{equation}
	1 = N(1) \neq N(aa^\dagger - a^\dagger a) = 0
\end{equation}
\begin{gather}
	\phi(T)\phi(0) = T\phi(T)\phi(0)\\
	\text{define $\psi(t) = \phi(-t)$}\nonumber\\
	= T\psi(-T)\psi(0)\\
	= \psi(0)\psi(-T)\\
	= \phi(0)\phi(T)
\end{gather}

The Time ordering Symbol and the box operator do not commute




%\feynmandiagram[horizontal=a to b]{
%i1 --[fermion] a -- [fermion] i2,
%a --[photon] b,
%f1 -- [fermion] b -- [fermion] f2,
%};

\end{document}

