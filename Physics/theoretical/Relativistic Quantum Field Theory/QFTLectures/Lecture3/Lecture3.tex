\documentclass[]{scrartcl}

\usepackage{\string~"/LaTeX/StylePackage"}

\title{Lecture 3}
\author{}
\date{\today}


\begin{document}

\maketitle
\newpage
\tableofcontents
\newpage

\section{Classical Klein-Gordon field} 

\begin{gather}
	L = \frac{1}{2}(\partial_\mu\phi)^2 - \frac{1}{2}m^2\phi^2\\
	\pi = \dot\phi, \;\; \mathcal{H} = \frac{1}{2}\dot\phi^2 + \frac{1}{2}(\nabla\phi)^2 + \frac{1}{2}m^2\phi^2\\
	\text{Klein-Gordon Equation: } (\partial^2 + m^2)\phi = 0\\
	\text{Solution: } \phi(x) = \int \frac{\text{d}^3p}{(2\pi)^3} \frac{1}{\sqrt{2\omega_p}}\left(a_p e^{-ipx} + \bar{a_p}e^{ipx}\right)
\end{gather}

\section{Quantized Klein-Gordon field}
Fields promoted to Operators

\begin{gather}
	\left[\phi(x), \pi(y)\right] = i\delta(x-y)\\
	\left[\phi(x), \phi(y)\right] = [\pi(x),\pi(y)] = 0
\end{gather}

We consider only equal-time commutators.

\begin{gather}
	\phi(x) \text{satisfies the Klein-Gordon equation}\\
	\phi(x) = \int\frac{\text{d}^3p}{(2\pi)^3}\frac{1}{\sqrt{2\omega_p}}\left[a_p e^{-ipx} + a_p^\dagger e^{ipx}\right]
\end{gather}
($\phi(x)$ is hermitian)
\begin{gather}
	= \int\frac{\text{d}^3 p}{(2\pi)^3}\frac{1}{\sqrt{2\omega_p}}[a_p e^{-i\omega_p t + ipx} + a^\dagger e^{i\omega_p t - ipx}]\\
	\pi(x) = \dot\phi(x) = (-i)\int\frac{\text d^3p}{(2\pi)^3}\sqrt{\frac{\omega}{2}}[a_p e^{-i\omega_p t + ipx} - a^\dagger e^{i\omega_p t - ipx}]
\end{gather}

we want the commutator relations of the $a_p$.
\begin{gather}
	[a_p, a_q^\dagger] = (2\pi)^3\delta(p-q)
	[a_p, a_q] = [a_p^\dagger, a_q^\dagger] = 0
\end{gather}
Assume this to be true and evaluate the previous commutator relations:
\begin{gather*}
	[\phi(x), \pi(y)] = \frac{-i}{2}\int\frac{\text d^3 p}{(2\pi)^3} \int \frac{\text d^3 q}{(2\pi)^3} \frac{\omega_p}{\omega_q}\left(-\underbrace{[a_p, a_q^\dagger]}_{(2\pi)^3\delta(p-q)}e^{i(px-qy)} + \underbrace{[a_p^\dagger, a_q]}_{-(2\pi)^3\delta(p-q)}e^{i(qy-px)}\right)\\
	= \frac{-i}{2}\int\frac{\text d^3p}{(2\pi)^3}(- e^{ip(x-y)} - e^{-ip(x-y)})\\
	= \frac{-i}{2}\int\frac{\text d^3p}{(2\pi)^3}(-2e^{ip(x-y)})\\
	= (-i)(-1)\delta(x-y)
\end{gather*}

Next task: calculating the hamiltonian.
\begin{gather}
	\mathcal{H} = \frac{1}{2}\pi^2 + \frac{1}{2} (\nabla\phi)^2 + \frac{1}{2}m^2\phi^2\text{: Hamiltonian Density}\\
	H = \int\text d^3x \mathcal{H} = \int \text d^3x \mathcal{H}\big\rvert_{t=0}\text{: no explicit t dependency in $L$.}\\
	\phi(x) = \int \frac{\text d^3 p}{(2\pi)^3}\frac{1}{\sqrt{2\omega_p}}[a_p + a_{-p}^\dagger]e^{ipx}\\
	\pi(x) = \int \frac{\text d^3p}{(2\pi)^3}\sqrt{\frac{\omega_p}{2}}[a_q - a_{-q}^\dagger]e^{iqx}\\
	H = \frac{1}{4}\int\text d^3x \int \frac{\text d^3p}{(2\pi)^3} \int \frac{\text d ^3q}{(2\pi)^3} \biggl[-\sqrt{\omega_p\omega_q}(a_p -a_{-p}^\dagger)(a_q - a_{-q}^\dagger)e^{i(p+q)x}\\ 
	+ \frac{1}{\sqrt{\omega_p\omega_q}}(ip)(iq)(a_p + a_{-p}^\dagger)(a_q + a_{-q}^\dagger)e^{i(p+q)x} \nonumber \\ 
	+ \frac{m^2}{\sqrt{\omega_p\omega_q}}(a_p + a_{-p}^\dagger)(a_q + a_{-q}^\dagger)e^{i(p+q)x}\biggr] \nonumber
\end{gather}
Now use Fubini to interchange the spatial and momentum integrals, which leads to delta functions $\delta(p + q)$.
\begin{gather}
	H = \frac{1}{4}\int \frac{\text d^3p}{(2\pi)^3}\biggl[-\omega_p (a_p - a_{-p}^\dagger)(a_{-p}- a_p^\dagger)\\
	+ \frac{p^2}{\omega_p}(a_p + a_{-p}^\dagger)(a_{-p} + a_p^\dagger) - \frac{m^2}{\omega_p}(\cdots)(\cdots)\biggr]\;\; \omega_p = p^2 + m^2 \nonumber\\
	H = \frac{1}{4}\int \frac{\text d^3p}{(2\pi)^3}-\omega_p (a_p - a_{-p}^\dagger)(a_{-p} - a_p^\dagger) + \omega_p (a_p + a_{-p}^\dagger)(a_{-p} + a_p^\dagger)\\
	= \frac{1}{4}\int \frac{\text d^3p}{(2\pi)^3} \omega_p (- a_p a_{-p} + a_pa_p^\dagger + a_{-p}^\dagger a_{-p} - a_{-p}^\dagger a_p^\dagger + a_p a_{-p} + a_pa_p^\dagger + a_{-p}^\dagger a_{-p} + a_{-p}^\dagger a_p^\dagger\\
	= \frac{1}{2}\int \frac{\text d^3p}{(2\pi)^3}\omega_p (a_p a_p^\dagger + a_p^\dagger a_p) \;\;\; [a_p, a_q^\dagger] = (2\pi)^3 \delta(p-q)\\
	= \int \frac{\text d^3p}{(2\pi)^3} \omega_p \biggr(a_p^\dagger a_p + \frac{1}{2}(2\pi)^3\delta(0)\biggl)
\end{gather}
We get an infinity in the Hamiltonian, by subtracting the infinity. Then
\begin{equation}
	H = \int \frac{\text d^3p}{(2\pi)^3}\omega_p(a_p^\dagger a_p)
\end{equation}
For the Quantum Mechanical Harmonic Oscillator we have the Hamiltonian
\begin{equation}
	H = \frac{p^2}{2} + 1/2\omega^2q^2 = \cdots = a^\dagger a + \frac{1}{2}
\end{equation}
We could have just as well written the Hamiltonian in the form
\begin{equation}
	\frac{1}{2}(\omega q - ip)(\omega q + ip) = a^\dagger a
\end{equation}
So we essentially picked a different point of view to write down the Hamiltonian. This also goes for our Infinite Hamiltonian.

We have: $[H, a_p^\dagger] = \int \frac{\text d^3p}{(2\pi)^3} \omega_q (a_q^\dagger a_q, a_p)$
\begin{gather}
	\int \frac{\text d^3p}{(2\pi)^3} \omega_q a_q^\dagger (2\pi)^3(q-p) = \omega_p a_p^\dagger\\
	\text{Similarly, } [H, a_p] = -\omega_p a_p
\end{gather}
Which are the same commutators as the ones in Quantum Mechanics. Then, we know that $a_p^\dagger$ and $a_p$ are ladder operators. And, every $p$-mode is a harmonic oscillator.

What "is" the state $a_p^\dagger \ket{0}$? We will calculate the Eigenvalue of the Hamiltonian of the state.
\begin{gather}
	Ha_p^\dagger \ket{0} = (a_p^\dagger H + \omega_p a_p^\dagger) \ket{0}\\
	= \omega_p a_p^\dagger \ket 0
\end{gather}
which means that our ket $a_p^\dagger\ket0$ has an Eigenvalue $\omega_p$ $(\hbar\omega_p)$.

The total momentum operator looks like
\begin{equation}
	P = -\int \text d^3x \pi\nabla\phi = \cdots = \int\frac{\text d^3p}{(2\pi)^3}pa_p^\dagger a_p
\end{equation}
so then, similarly $P a_p^\dagger \ket 0 = p a_p^\dagger \ket 0$.\\
Then, $a_p^\dagger$ is a single excitation of the harmonic oscillator associated with mode $p$, with Energy $\omega_p$, and the momentum $p$. Called a \textbf{particle}.

Then, a particle is an excitation of a quantum field.\\
We have a two particle system $a_p^\dagger a_q^\dagger \ket{0}$. And an n-particle state $(a_p^\dagger)^n\ket 0$, these are Bosons. Later on, for fermions $(a_p^\dagger)^2 = 0$.

To recall, we have the Klein Gordon field, which is the solution to the KG Equation. We define the commutator relations that we want. It turns out that $a_p$ and $a_p^\dagger$ are ladder operators. The excitation of the empty space is a particle. A particle is an infinite plane wave, which means that we need to overlap several particles to get a localized particle. 

next Lecture: Lorentz Invariance.

\section{Lorentz-Transformation}

\begin{equation}
	x^\mu \mapsto x'^\mu = \Lambda^\mu_\nu x^\nu
\end{equation}
we want a transformation such that
\begin{equation}
	\text ds^2 = \text d (x^0)^2 - \text d x^2 = g_{\mu\nu}\text dx^\mu \text dx^\nu
\end{equation}
is invariant.
\begin{gather}
	g_{\mu\nu} = \text d x^\mu \text d x^\nu = g_{\rho\sigma}\text d x'^\rho \text d x'^\sigma\\
	= g_{\rho\sigma}\Lambda^\rho_\mu \text d x^\mu \Lambda^\sigma_\nu \text d x^\nu\\
	\Rightarrow g_{\mu\nu} = g_{\rho\sigma}\Lambda^\rho_\mu \Lambda^\sigma_\nu
\end{gather}
\end{document}

