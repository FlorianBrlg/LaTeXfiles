\documentclass[]{scrartcl}

\usepackage{\string~"/LaTeX/StylePackage"}

\title{}
\author{}
\date{\today}


\begin{document}

\maketitle
\newpage
\tableofcontents
\newpage

Why QFT?

\begin{itemize}
	\item nonrelativistic Quantum Mechanics is Noncausal

	\item Quantum Mechanics cannot describe generation/annihilation of particles

	\item Even in low-energy Physics, QFT is used to describe the Photon field
\end{itemize}

\section{Classical field theory}

In classical mechanics:
 \begin{equation}
 	s = \int L(q_i, \dot{q_i}) \text{d}t
 \end{equation}
It is required that the action is stationary, $\delta s = 0$, from which the Euler Lagrange equations are derived.

\subsection{Example: String}

Abbildung 1.2

we consider longitudal oscillation. And use the discretization:

Abbildung 1.2

We then have the Lagrangian
\begin{equation}
	L = T-V = \sum_i \left(\frac{1}{2}m\dot{q_i}^2 - \frac{1}{2} k(q_{i+1} - q_i)^2\right)
\end{equation}

For field theory, $q$ will become the field, and $x$ will be the discretization $i$.

\begin{equation}
	L = \sum_i a\left(\frac{1}{2}\frac{m}{a}\dot{q_i}^2 - \frac{1}{2}ka\left(\frac{q_{i+1} - q_i}{a}\right)^2\right)
\end{equation}
$a\rightarrow 0$.
\begin{equation}
	L = \int \text{d}x \underbrace{\left[\frac{1}{2}\mu \dot{q_i}^2 - \frac{1}{2}Y\left(\pdv{q}{x}\right)^2\right]}_{\text{Lagrange Density}}
\end{equation}

$i \mapsto x$, and $q_i \mapsto \varphi$.

Abbildung 2

\section{Lagrangian field theory}
\begin{equation}
s = \int \text d t L = \int \text{d}^3 x \int \text{d}t L = \int \text{d}^4x L
\end{equation}
where we have
\begin{equation}
	L = L(\varphi, \partial_\mu \varphi)
\end{equation}
and we use the principle of least action, such that $\delta s = 0$.

\begin{gather}
\delta s = \int \text{d}^4x \left(\pdv{L}{\varphi}\delta\varphi + \pdv{L}{(\partial_\mu \varphi)}\delta(\partial_\mu \varphi)\right)\\
= \int\text{d}^4x\,\partial_\mu\underbrace{\left(\pdv{L}{(\partial_\mu \varphi)}\delta\varphi\right)}_{K^\mu} + \int\text{d}^4x\left(\pdv{L}{\varphi} - \partial_\mu\left(\pdv{L}{(\partial_\mu \varphi)}\right)\right)\delta\varphi\\
\int\text{d}^4x \partial_\mu K^\mu = \int_{\text{bounding surface, }B} K^\mu \text{d}S_\mu\\
\text{Remember: }\int_V\text{d}^3x \nabla A = \oint_S dS A
\end{gather}

and we require the action to be zero at the bounding surface, so for all points $x\in B$, $\delta\phi(x) = 0$.
\begin{equation}
	\text{Then: }\int \text{d}S_\mu K^\mu = 0.
\end{equation}
And then
\begin{equation}
	\pdv{L}{\varphi} - \partial_\mu\left(\pdv{L}{(\partial_\mu\varphi)}\right) = 0
\end{equation}
Which is the Euler-Lagrange equation, as desired. This follows from the fundamental theorem of variational calculus and $Eq.8$.

\subsection{Example: String}
$q = \varphi$.
\begin{gather}
	\pdv{L}{q}= 0\;,\;\pdv{L}{\dot{q}} = \mu\dot{q}\;,\;\pdv{L}{(\pdv{q}{x})} = -Y\pdv{q}{x}\\
	\pdv{L}{q} - \partial_0 \pdv{L}{\partial_0 q} - \partial_i\left(\pdv{L}{(\partial_i q)}\right) = 0\\
	-\mu\ddot{q} + Y \pdv[2]{q}{x} = 0\;\;\;\text{Wave Equation}
\end{gather}

\subsubsection{Four-Vector}
$A^\mu$ is a four-vector if it transforms as $x^\mu$ under Lorentz Transformations.

\section{Hamiltonian Field Theory}

Classical mechanics:
\begin{gather}
	p_i = \pdv{L}{\dot{q_i}}\text{: conjugated momentum}\\
	H = \sum_i p_i\dot{q_i} - L = E
\end{gather}
For the string, this means that
\begin{equation}
	p_i = m\dot{q_i}\;,\; H = \sum_i \left(\frac{1}{2}m\dot{q_i}^2 + \frac{1}{2}k(q_{i+1} - q_i)^2\right)
\end{equation}
and with $a\rightarrow 0$
\begin{equation}
	H = \int\text d x \left(\frac{1}{2}\mu\dot{q_i}^2 + \frac{1}{2}Y\left(\pdv{L}{x}\right)^2\right)
\end{equation}

Classical field theory:
\begin{gather}
	\pi = \pdv{L}{\dot\varphi}\text{: conjugated momentum density}\\
	\mathcal{H} = \pi\dot\varphi - L\;,\; H = \int \text d x \mathcal{H}
\end{gather}

\section{Klein-Gordon field}

\begin{equation}
	\frac{1}{2}\underbrace{(\partial_\mu \varphi)(\partial^\mu\varphi)}_{\text{P\&S: } (\partial_\mu\varphi)^2} - \frac{1}{2}m^2\phi^2
\end{equation}
Euler-Equation:
\begin{equation}
	-m^2\varphi - \partial_\mu\partial^\mu \varphi = (m^2 + \partial_\mu\partial^\mu)\varphi = 0
\end{equation}
Then we have
\begin{equation}
	\pdv[2]{\varphi}{t} - \nabla^2\varphi + m^2\varphi = 0
\end{equation}
and for the Hamiltonian field
\begin{gather}
	\pi = \pdv{L}{\dot\varphi} = \dot\varphi\\
	\mathcal{H} = \dot\varphi^2 - L = \frac{1}{2}(\dot\varphi^2 + (\nabla\varphi)^2) + \frac{1}{2}m^2\varphi^2
\end{gather}
Which is not Lorentz invariant, as you give Energy into the system.

\section{Noether Theorem}

If you have a continuous symmetry, then you get a conservation law.
\begin{itemize}
	\item Translational Symmetry: conservation of momentum
	\item Rotational Symmetry: conservation of angular momentum
	\item Time Symmetry: conservation of energy
\end{itemize}

\subsubsection{Symmetry Definition}
A symmetry is a transformation such that the equation of motion is unchanged.

\begin{equation}
	L \mapsto L + \alpha \partial_\mu J^\mu
\end{equation}
We start with transforming $L$ to $L$ with a four-divergence. This does not change the action, as a four-divergence is 0 at the boundary.

\end{document}

