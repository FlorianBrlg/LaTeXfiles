\documentclass[]{scrartcl}

\usepackage{\string~"/LaTeX/StylePackage"}

\title{QFT - Lecture 13}
\author{}
\date{7.10.2024}


\begin{document}

\maketitle
\newpage
\tableofcontents
\newpage

\section{Interacting fields} 

1)
$$
\varphi^4 \text{-Theory: } \mathcal{L} = \frac{1}{2}\partial_\mu\varphi\partial^\mu\varphi - \frac{1}{2}m^2\varphi^2 - \frac{\lambda}{4!}\varphi^4
$$
\begin{itemize}
	\item Self interaction theory
	\item Simplest interaction theory (Interaction between Fourier components)
	\item Contains all essential features of an interaction theory
	\item Not possible to solve analytically
	\item Relevance to the Standard model and in statistical mechanics and solid state physics
\end{itemize}
2)
$$
\text{Yukawa-Theory: } \mathcal{L}_\text{Yukawa} = L_{\text{Dirac}} + L_\text{KG} - g\bar\psi\psi\varphi
$$
\begin{itemize}
	\item Very similar to QED, but simpler
\end{itemize}
3)
$$
\text{QED: } \mathcal{L}_\text{QED} = \mathcal{L}_\text{Maxwell} + \mathcal{L}_\text{Dirac} - \underbrace{e\bar\psi\gamma^\mu\psi A_\mu}_{j^\mu A_\mu}
$$
$$
= \mathcal L_\text{Maxwell} + \bar\psi(i\slashed\partial - m)\psi - \underbrace{e\bar\psi\gamma^\mu\psi A_\mu}_{e\bar\psi \slashed A \psi}
$$
$$
= \mathcal L_\text{Maxwell} + \bar\psi(i\slashed D -m)\psi,\;\; \slashed D = \slashed\partial + ie \slashed A,\;\text{ Gauge covariant Derivative}
$$
\begin{itemize}
	\item explains almost all phenomena down to $10^{-15}m$
	\item The QED Lagrangian is invariant under gauge transformation $A_\mu \mapsto A_\mu - \frac{1}{e}\partial_\mu \alpha(x)$ and $\psi \mapsto e^{i\alpha(x)}\psi$ together
\end{itemize}

\section{The Interaction Picture}
\begin{itemize}
	\item Define a $t_0$ where all pictures coincide.
	\item Schrödinger Picture:
		\begin{itemize}
			\item State kets evolve in time $\ket{\psi(t)} = e^{iH(t-t_0)}\ket{\psi(t_0)}$
			\item Operators $A(t_0)$ independent of time
		\end{itemize}
	\item Heisenberg Picture:
		\begin{itemize}
			\item State kets $\ket{\psi(t_0}$ are indepdentent of time
			\item Operators evolve in time $A(t) = e^{iH(t-t_0)}A(t_0)e^{-iH(t-t_0)}$
		\end{itemize}
	\item Interaction Picture:
		\begin{itemize}
			\item $A(t) := e^{iH_0(t-t_0)}A(t_0)e^{-iH_0(t-t_0)}$ where we have $H = H_0 + H_{\text{interacting}}$
			\item $\ket{\psi(t)} := e^{iH_0(t-t_0)}e^{-iH(t-t_0)}\ket{\psi(t_0)}$
		\end{itemize}
\end{itemize}

Calculating the expectation values:
\begin{gather}
	\langle A \rangle = \bra{\psi(t)} A(t) \ket{\psi(t)} = \cdots \\
	= \bra{\psi(t_0)} e^{iH(t-t_0)} A(t_0) e^{-iH(t-t_0)}\ket{\psi(t_0)}
\end{gather}
Showing that the definitions are equivalent physically.

We define the Operator:
$$
U(t,t_0) = e^{iH_0(t-t_0)}e^{-iH(t-t_0)}
$$
Some properties we want:
\begin{itemize}
	\item $U(t_3,t_2)U(t_2,t_1) = U(t_3,t_1)$
	\item $U(t_2,t_1) = U^\dagger(t_1,t_2)$
	\item $U(t,t')$ General
\end{itemize}

We want an expression for $U(t,t') = U(t,t_0) U(t_0,t')$
\begin{gather}
	U(t,t_0)U^\dagger(t',t_0) = e^{iH_0(t-t_0)}e^{-iH(t-t_0)}e^{iH(t'-t_0)}e^{-iH_0(t'-t_0)}\\
	= e^{iH_0(t-t_0)}e^{iH(t'-t)}e^{-iH_0(t'-t_0)}
\end{gather}


\section{Perturbation Theory}
Goal: Calculating $\bra\Omega T\phi(x)\phi(y)\ket\Omega$ in $\phi^4$-Theory
$$
H = H_0 + \int \text d^3x \frac{\lambda}{4!}\phi^4
$$
with $H_0$ being the KG Hamiltonian.\\
Hamiltonian in the interaction picture:
\begin{gather}
	e^{iH_0(t-t_0)}(H_0 + H_\text{int})e^{-iH_0(t-t_0)}\\
	= H_0 + H_I(t),\;\; H_I = e^{iH_0(t-t_0)}H_\text{int}e^{-iH_0(t-t_0)}\\
	\phi_I(t) = e^{iH_0(t-t_0)}\phi(t_0)e^{-iH_0(t-t_0)}\\
		= e^{iH_0(t-t_0)}e^{-iH(t-t_0)}\phi(t)e^{iH(t-t_0)}e^{-iH_0(t-t_0)}\\
		= U(t,t_0)\phi(t)U^\dagger(t,t_0)\\
		\Rightarrow \phi(t) = U^\dagger(t,t_0)\phi_I(t)U(t,t_0)
\end{gather}

Find a useful expression for $U(t,t')$.
\begin{gather}
	i\pdv{U(t,t')}{t} = e^{iH_0(t-t_0)}\underbrace{(H-H_0)}_{H_\text{int}}\left(e^{-iH_0(t-t_0)}e^{iH_0(t-t_0)}\right)e^{iH(t'-t)}e^{-iH_0(t-t_0)}\\
	= H_I(t) U(t,t')
\end{gather}
Solution through Dyson series:
\begin{gather}
	U(t,t') = 1 + (-i)\int_{t'}^t \text dt_1 H_I(t_1) + (-i)^2 \int_{t'}^t\int_{t'}^{t_1}\text dt_1 \text dt_2 H_I(t_1)H_I(t_2) + \cdots\\
	\pdv{U}{t_1} = (-i)H_I(t_1) + (-i)^2 H_I(t_1)\int_{t'}^t \text d t_2 H_I(t_2) + \cdots
\end{gather}
Useful rewrite:
\begin{gather}
	U(t,t') = 1 + (-1) \int_{t'}^t \text dt_1 H_I(t_1) + \frac{(-i)^2}{2}\int_{t'}^t\int_{t'}^{t_1}\text dt_1 \text dt_2 T\{H_I(t_1)H_I(t_2)\} + \cdots\\
	= T\left\{\exp\left[-i\int_{t'}^t \text dt''H_I(t'')\right]\right\}
\end{gather}
Why is this more convenient?
\begin{equation}
	H_I(t) = \int \text d^3x \frac{\lambda}{4!}\varphi_I^4
\end{equation}
We now have $\phi(t)$, $\phi_I$ and $U(t,t')$.

Next time: what is $\ket\Omega$?
\end{document}

