!%TeX program = lualatex
\documentclass[]{scrartcl}

\usepackage{\string~"/LaTeX/StylePackage"}
\usepackage{iftex}
\title{QFT - Lecture 17}
\author{}
\date{21.10.2024}


\begin{document}

I have been compiled with\dots{\ttfamily
\ifluatex
    luatex
\else\ifxetex
    xetex
\else\ifpdftex
    pdftex
\else
    NO IDEA!
\fi\fi\fi}


\maketitle
\newpage
\tableofcontents
\newpage

\section{LSZ Reduction} 

\begin{gather}
	a_p^\psi = i\int \text d^3x \bar\Psi_p(x) \partial_0 \phi(x)\\
	\Psi_p(x) = \int\frac{\text d^3p}{(2\pi)^3}\frac{\psi(k-p)}{\sqrt{2E_p}}e^{-ikx}\\
	k^0 = E_k = \sqrt{k^2 + m^2}
\end{gather}
Wavepacket in and out states
\begin{gather}
	\ket{k_1,k_2}_{\text{in}} := \ket{\text{int}}= \prod_i \sqrt{\frac{2E_{k_i}}{Z}}a_{k_1}^{\psi\dagger}(-\infty)a_{k_2}^{\psi\dagger}(-\infty)\cdots\ket\Omega\\
	\ket{p_1,p_2}_\text{out} := \ket{\text{out}} = \prod_j \sqrt\frac{2E_{p_j}}{Z} a_{p_1}^{\psi\dagger}(\infty)a_{p_2}^{\psi\dagger}(\infty)\cdots\ket\Omega
\end{gather}
with renormalization constants
\begin{gather}
	Z_\lambda = |\bra\Omega \phi(0) \ket{\lambda_{p=0}}|^2\\
	Z = |\bra\Omega \phi(0) \ket{p=0}|^2
\end{gather}
we consider the S matrix, which looks like
\begin{gather}
	\braket{\text{out}|\text{int}} = \bra{p_1,p_2,\cdots}S\ket{k_1,k_2,\cdots}
\end{gather}
We want to get an expression that uses the fields, instead of the ladder operators, as we know how to do calculations with them
\begin{gather}
	\bra\Omega T\phi(x)\phi(y) \ket\Omega
\end{gather}
We will define from now on that $\prod_i \sqrt{2E_{k_i}/Z} = A$.

\begin{gather}
	\braket{\text{int}|\text{out}} = A\bra{\text{out}}a_{k_1}^{\psi\dagger}(-\infty)a_{k_2}^{\psi\dagger}(-\infty)\ket\Omega\\
	= A\bra{\text{out}}\left((a_{k_1}^{\psi\dagger}(-\infty) - a_{k_1}^{\psi\dagger}(\infty)\right) a_{k_2}^{\psi\dagger}(-\infty) \ket\Omega\\
	\text{Holding as far as $k_1 \neq p_1,p_2,..$}
\end{gather}
if $k_1 \neq p_1,p_2$, then they go in different directions. No matter how close they are at the beginning, they will be spacially separated at $\infty$. This means that you can commute the ladder operators which kills the vacuum, resulting in a zero, which is why we can add the term. We will now insert the definitions of the ladder operators with limits.
\begin{gather}
	iA \left[\lim_{t_1 \rightarrow \infty} - \lim_{t_1\rightarrow -\infty}\right] \int\text d^3x \Psi_{k_1}(x_1)\overleftrightarrow\partial_0^{x_1}\bra{\text{out}} \phi(x_1)a_{k_2}^{\psi\dagger}(-\infty)\ket\Omega\\
	\text{we use the identity}\nonumber\\
	\left[\lim_{t\rightarrow\infty} - \lim_{t\rightarrow-\infty}\right] \int\text d^3x f(x)\overleftrightarrow\partial_0 g(x) = \int\text d^4x\partial_0\left(f(x)\overleftrightarrow\partial_0 g(x)\right)\\
	= \int\text{d}^4x \left[f(x)\partial_0^2g(x) - \left(\partial_0^2 f(x)\right) g(x)\right],\;\text{$f(x)$ statisfies KG-eq}\\
	= \int\text d^4x \left[f(x)\partial_0^2g(x) - \left((m^2 - \nabla^2)f(x)\right)g(x)\right]\\
	\text{and we use integration by parts, surface term goes to 0 if $f(x)\rightarrow0$.}\nonumber\\
	\int\text d^4x f(x)\left(\partial_\mu\partial^\mu + m^2\right)g(x)
\end{gather}
And we use this property on the previous Integral
\begin{gather}
	(13) = iA\int\text d^4x \Psi_1(x_1)\left((\partial_\mu\partial^\mu)_{x_1}+m^2\right)\bra{\text{out}}\phi(x_1) a_{k_2}^{\psi\dagger}(-\infty)\ket\Omega
\end{gather}
Next step. Result:
\begin{gather}
	\braket{\text{out}|\text{in}} = \prod_i i\sqrt{\frac{2E_{k_i}}{Z}}\int \text d^4x_i \Psi_{k_i}(x_i) \left((\partial_\mu\partial^\mu)_{x_i} + m^2\right)\nonumber\\
\prod_j i\sqrt{\frac{2E_{p_j}}{Z}}\int \text d^4y_j \bar\Psi_{p_j}(y_j)\left((\partial_\mu\partial^\mu)_{y_j} + m^2\right) \bra\Omega T\phi(x_1)\cdots\phi(y_1)\cdots\ket\Omega
\end{gather}
Take plane-wave limit, $\psi(k) \rightarrow (2\pi)^3\delta(k)$ which then results in
$$
\Psi_{p}(x) \rightarrow \frac{1}{\sqrt{2E_p}}e^{-ipx}
$$
And then we get
\begin{gather}
	\braket{\text{out}|\text{in}} = \prod_i \frac{1}{i\sqrt Z}\left(k_i^2 - m^2\right)\int\text d^4x_i e^{-ik_ix_i} \prod_j\frac{1}{i\sqrt Z}\left(p_j^2 - m^2\right) \int \text d^4y_j e^{ip_jy_j}\\
	\underbrace{\bra\Omega T\phi(x_1)\cdots\phi(y_1)\cdots\ket\Omega}_{\text{sum of connected diagrams}}
\end{gather}

Fourier transforming the diagrams can be done by omitting the exponentials and Integrals of external points.

Consider Fig1.

Fourier Transforming:
\begin{gather}
	-i\lambda (2\pi)^4 \delta(p_1 + p_2 - k_1 - k_2) D_F(k_1)D_F(k_2)D_F(p_1)D_F(p_2)
\end{gather}

it is then convenient to define an S Matrix
\begin{gather}
	\text{Define: } i\mathcal{M}:\;\; \braket{\text{out}|\text{in}} = (2\pi)^4 \delta\left(\sum p_j - \sum k_i\right) i\mathcal{M}(k_1,k_2,\cdots p_1,\cdots)\\
	i\mathcal{M} = \prod_i \frac{1}{i\sqrt Z}\left(k_i^2 - m^2\right) \prod_j \frac{1}{\sqrt Z}\left(p_j^2 -m^2\right)\sum\text{connected diagrams}
\end{gather}
where the connected diagrams are with the Feynman rules in momentum space, but not with exponentials and momentum integrals for external lines.


\end{document}

