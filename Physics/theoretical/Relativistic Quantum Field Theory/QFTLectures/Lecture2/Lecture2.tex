\documentclass[]{scrartcl}

\usepackage{\string~"/LaTeX/StylePackage"}

\title{QFT - Lecture 2}
\author{}
\date{21.8.2024}


\begin{document}

\maketitle
\newpage
\tableofcontents
\newpage

 \section{Noether's Theorem}

A continuous symmetry leads to a conservation law. A symmetry is a Transformation that leaves the equation of motion invariant.

Infinitesimal transformation:
\begin{gather}
	\varphi(x) \mapsto \Phi + \alpha\Delta\varphi(x)\\
	L(x) \mapsto L(x) + \alpha\Delta L(x)
\end{gather}
Where $\alpha$ is infinitesimal\\
Assume, that
\begin{equation}
	L(x)\mapsto L(x) + \alpha\partial_\mu J^\mu\;\;\text{for some }J^\mu
\end{equation}
Meaning we assume that $\Delta\alpha$ can be written as a four-divergence.\\
So we have the action
\begin{equation}
	s = \int\text d ^4x L + \alpha \int_\text{Surface}\text{d} S_\mu J^\mu
\end{equation}
And looking at the change of the action, the first integral stays the same and the Surface integral is 0, meaning the equation of motion stays the same.
\begin{gather}
	\alpha\Delta L = \pdv{L}{\varphi}\Delta\varphi + \pdv{L}{(\partial_\mu\varphi)}\underbrace{\Delta(\alpha\partial_\mu\varphi)}_{= \partial_\mu(\alpha\Delta\varphi)}\\
	= \partial_\mu\left(\pdv{L}{(\partial_\mu\varphi)}\alpha\Delta\varphi\right) + \underbrace{\left(\pdv{L}{\varphi}-\partial_\mu\pdv{L}{(\partial_\mu\varphi)}\right)}_{0\text{ by Euler-Lagrange}}\alpha\Delta\varphi
\end{gather}
From our assumption we can then gather, that
\begin{equation}
	\partial_\mu\left(\pdv{L}{(\partial_\mu\varphi)}\Delta\varphi - J^\mu\right) = 0
\end{equation}

we then gather, that
\begin{equation}
	\partial_mu j^\mu = 0,\;\;\;\; j^\mu = \pdv{L}{(\partial_\mu\varphi)}\Delta\varphi - J^\mu
\end{equation}
If the time derivative is integrated over all space, then it can be seen that it is conserved.\\
$j^\mu$ is the Noether current.
\subsection{Example:}
\begin{equation}
	L = (\partial_\mu\varphi)(\partial^\mu\varphi*) - m\varphi\varphi*
\end{equation}
The Complex Klein Gordon field.\\
Consider the transformation
\begin{equation}
	\varphi\mapsto e^{i\alpha}\varphi,\;\;\; \alpha=const
\end{equation}
This is a Symmetry transformation, as $L \mapsto L$. That means, $J^\mu = 0$.\\
We treat $\varphi,\,\varphi*$ as independent fields. 

\subsubsection{More than one field}
If there is more than one field, $\varphi_r$, then
\begin{equation}
	L = L(\varphi_1, \varphi_2,..., \partial_\mu\varphi_1, \partial_\mu\varphi_2,...)
\end{equation}
There is one equation of motion for each field, that is because we can vary every field by itself, whose actions must each be minimal.

Then, the Noether charge will look like
\begin{equation}
	\sum_r \pdv{L}{(\partial_\mu\varphi_r)}\Delta\varphi_r - J^\mu
\end{equation}

Then,
\begin{align}
	j^\mu &= \pdv{L}{(\partial_\mu\varphi)}\Delta\varphi + \pdv{L}{(\partial_\mu\bar\varphi)}\Delta\bar\varphi\\
	\varphi &\mapsto e^{i\alpha}\varphi = (1+i\alpha)\varphi \Longrightarrow \alpha\Delta\varphi = i\alpha\varphi,\; \alpha\Delta\bar\varphi = -i\alpha\bar\varphi\\
	j^\mu &= (\partial^\mu\bar\varphi)i\varphi + (\partial^\mu\varphi)(-i\bar\varphi)
\end{align}

\subsection{Example:}
Translation: $x^\mu \mapsto x^\mu + a^\mu$\\
$\varphi(x) \mapsto \varphi(x+a) = \varphi(x) + a^\mu\partial_\mu\varphi(x)$ As $a$ is infinitesimal.\\
$L \mapsto L(x + a) = L(x) + a^\mu\partial_\mu L(x) = L(x) + a^\nu\partial_\mu(\delta^\mu_\nu L(x))$\\
For $a^0 = \alpha,\,a^i=0$:	$\alpha\partial_\mu\underbrace{(\delta^\mu_0 L(x))}_{J^\mu}$

and the Noether current: 
\begin{equation}
j^\mu = \pdv{L}{(\partial_\mu\varphi}\underbrace{\Delta\phi}_{\dot{\varphi}} - \delta^\mu_0 L(x)
\end{equation}
Then, for a general translation
\begin{equation}
	T^\mu_\nu = \pdv{L}{(\partial_\mu\varphi)}\partial_\nu\varphi - \delta^\mu_\nu L(x)
\end{equation}
This is the energy-momentum Tensor\\
Through the Noether theorem, we then have, for each $\nu$, the conserved current
\begin{equation}
	\partial_\mu T^\mu_\nu = 0
\end{equation}

If we raise the index of the tensor,
\begin{equation}
	T^{\mu\nu} = \pdv{L}{(\partial_\mu\varphi)}\partial^\nu\varphi - g^{\mu\nu}L(x)
\end{equation}
Then we get
\begin{equation}
	T^{00} = \pdv{L}{\dot\varphi}\varphi - L(x) = \pi\dot\varphi - L = \mathcal{H} = \text{energy}
\end{equation}
\begin{equation}
	T^{0i} = \pdv{L}{\partial_0\varphi}\partial^i\varphi = \pi\partial^i\varphi = -\pi\partial_i\varphi
\end{equation}
Define $P^i$ as
\begin{equation}
	P^i = \int \text{d}^3x T^{0i}: \text{ physical momentum carried by the field}
\end{equation}
Is this the momentum we are used to?

\section{Classical, real Klein-Gordon field}
\begin{equation}
	L = \frac{1}{2}(\partial_\mu\varphi)^2 - \frac{1}{2}m^2\varphi^2
\end{equation}
\begin{equation}
	\pi = \pdv{L}{\dot\varphi} = \dot\varphi
\end{equation}
The physical momentum and the conjugate momentum density doesn't have to be equal.
\begin{equation}
	\mathcal{H} = \pi\dot\varphi - L = \dot\varphi^2 - L = \frac{1}{2}\dot\varphi^2 + \frac{1}{2}(\nabla\varphi)^2 + \frac{1}{2}m^2\varphi^2
\end{equation}
And the Klein-Gordon equation
\begin{equation}
	(\partial_\mu\partial^\mu + m^2)\varphi = 0 = (\partial_t - \nabla^2 + m^2)\varphi
\end{equation}
We fourier transform our equation in order to get a representation we can work with better.
\begin{equation}
	\varphi(x,t) = \int \frac{\text{d}^3p}{(2\pi)^3}e^{ipx}\cdot\underbrace{\varphi(p,t)}_{\text{Fourier Transform}}
\end{equation}
Consider the symmetry, that $\varphi$ is real.
\begin{equation}
	\int \frac{\text{d}^3p}{(2\pi)^3}e^{ipx}\varphi(p,t) = \int \frac{\text{d}^3p}{(2\pi)^3}e^{-ipx}\bar\varphi(p,t)
\end{equation}
Then, through a coordinate transformation $p\mapsto -p$,
\begin{equation}
	= \int \frac{\text{d}^3p}{(2\pi)^3}e^{ipx}\bar\varphi(-p,t)
\end{equation}
So we have the symmetry
\begin{equation}
	\varphi(p,t) = \bar\varphi(-p,t),\;\; \forall p,t
\end{equation}
And we return to the Klein-Gordon equation
\begin{gather}
	(\partial_t^2 - (ip)^2 + m^2)\varphi(p,t) = 0\\
	(\partial_t^2 + \omega_p^2)\varphi(p,t) = 0,\;\;\omega_p^2 = m^2+p^2
\end{gather}
And we get the solution
\begin{equation}
	\varphi(p,t) = \frac{1}{\sqrt{2\omega_p}}\left(a_pe^{-i\omega_p t} + b_p e^{i\omega_p t}\right)
\end{equation}
Through the boundary condition, $\varphi(p,t) = \bar\varphi(-p,t)$ and $t = 0$ we also get
\begin{equation}
	\bar a_{-p} + \bar b_{-p} = a_p + b_p
\end{equation}
Then, this must also be the same after a time derivative,
\begin{gather}
	\bar a_{-p} - \bar b_{-p} = - a_p + b_p
	\Rightarrow b_p = \bar a_{-p}
\end{gather}
and then the full solution
\begin{equation}
	\varphi(x,t) = \int \frac{\text{d}^3p}{(2\pi)^3} e^{ipx}\frac{1}{\sqrt{2\omega_p}}\left(a_pe^{-i\omega_p t} + \bar a_{-p} e^{i\omega_p t}\right)
\end{equation}
\begin{equation}
	\int \frac{\text{d}^3 p}{(2\pi)^3}\frac{1}{\sqrt{2\omega_p}}\left(a_p e{-i\omega_p t + ipx} + \bar a_p e^{i\omega_p t - ipx}\right)
\end{equation}
and now change $p = (\omega_p, p)$ and $x = (t, x)$. So $p$ and $x$ are four vectors.
\begin{equation}
	\varphi(x) = \int \frac{\text{d}^3 p}{(2\pi)^3} \frac{1}{\sqrt{2\omega_p}}\left(a_p e^{-ipx} + \bar a_p e^{ipx}\right)
\end{equation}

For quantization we require, that the commutator
\begin{equation}
	[q_i, p_j] = i\hbar\delta_{ij} = i\delta_{ij}
\end{equation}
and because of fields,
\begin{equation}
	[\varphi(x), \pi(y)] = i\delta(x-y)
\end{equation}


\end{document}

