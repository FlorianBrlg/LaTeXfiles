\documentclass[]{scrartcl}

\usepackage{\string~"/LaTeX/StylePackage"}

\title{QFT - Lecture 25}
\author{}
\date{20.11.2024}


\begin{document}

\maketitle
\newpage
\tableofcontents
\newpage

\section{Renormalization of electric charge} 

Consider corrections to photon propagator

Vacuum polatization diagram: Fig1

\begin{gather}
	= (-ie)^2 (-1) \int\frac{\text d^4k}{(2\pi)^4}\text{tr}\left[\gamma^\mu \frac{i(\slashed k + m)}{k^2 - m^2 + i\epsilon} \gamma^\nu \frac{i(\slashed k + \slashed q + m)}{(k+q)^2 - m^2 + i\epsilon}\right]\\
	= i \Pi_2^{\mu\nu}(q)
\end{gather}
where we have (-1) and the trace because of the fermion loop.

More generally: Fig 2

$\Pi^{\mu\nu}(q)$ is a tensor. It is comprised of the tensors that are included in the calculation. Then
\begin{gather}
	\Pi^{\mu\nu}(q) = Cq^\mu q^\nu + D g^{\mu\nu}\text{, by Lorentz invariance}
\end{gather}
We use Ward's Identity:
\begin{gather}
	q_\mu \Pi^{\mu\nu} = 0
\end{gather}
Which we apply to eq3
\begin{gather}
	\Rightarrow Cq^2q^\nu + Dq\nu = 0\\
	\Rightarrow (Cq^2 + D) q^\nu =  0\\
	\Rightarrow Cq^2 + D = 0\\
	\Rightarrow D = -Cq^2
\end{gather}
Which we substitute into the previous equation
\begin{gather}
	\Pi^{\mu\nu}(q) = C(q^\mu q^\nu - q^2 g^{\mu\nu}\\
	\text{and we redefine $-C = \Pi(q^2)$.} \nonumber\\
	= (q^2 g^{\mu\nu} - q^\mu q^\nu)\Pi(q^2)
\end{gather}

\section{Sum of all correction to the poton propagator}

denoted by Fig 3

\begin{gather}
	= \text{Fig 4}
\end{gather}

Consider a low $q^2$ process $(q^2 << m^2)$:

Fig5

What if $q^2$ is not small?

\begin{gather}
	e_0 \rightarrow \frac{e_0}{\sqrt{1-\Pi(q^2)}} = e\frac{\sqrt{1-\Pi(0)}}{\sqrt{1-\Pi(q^2)}}
\end{gather}
The apparent charge is dependent on $q^2$.

That means there is an Effective potential from an electron that is 
\begin{gather}
	V(r) = -\frac{\alpha}{r}\left(1 + \frac{\alpha}{4\sqrt\pi}\frac{e^{-2mr}}{(mr)^{3/2}} + \cdots\right)
\end{gather}

Plotting this, with $V(r) = \frac{-\alpha_{eff}(r)}{r}$: Fig6

When the elctrons are closer together, they have a stronger attraction. Fig7
\end{document}

