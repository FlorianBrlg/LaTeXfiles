\documentclass[]{scrartcl}

\usepackage{\string~"/LaTeX/StylePackage"}

\title{QFT - Lecture9}
\author{}
\date{16.9.2024}


\begin{document}

\maketitle
\newpage
\tableofcontents
\newpage

\section{Functions of matrices}
1)\\
$A$ is the square matrix.\\
$fx) = \sum_{n}c_n x^n$. Then
$$
f(A) = \sum_{n} c_n A^n.
$$
assuming that the Taylor expansion exists, and converges.

2)\\
Now, assuming that $A$ is normal, meaning that $A^\dagger A = AA^\dagger$. Then
$$
A = U \Lambda U^\dagger
$$
then we can define
$$
f(A) = Uf(\Lambda)U^\dagger,\;\; f(\Lambda) = 
\begin{pmatrix}
	f(\lambda_1) & 0 & \cdots\\
	0 & f(\lambda_2) & \cdots\\
	\vdots & \cdots & 0
\end{pmatrix}
$$
When 1) and 2) both exist:
\begin{gather}
	f_1 = \sum c_n A^n = \sum c_n (U\Lambda U^\dagger)^n = \sum_n c_n U\Lambda^n U^\dagger\\
	= U \sum c_n A^n U^\dagger = U f(\Lambda) U ^\dagger = f_2
\end{gather}
If normal $A$ and $B$ commute, then $\sqrt{AB} = \sqrt A \sqrt B$.

\textit{Proof:}\\
\begin{gather}
	[A,B] = 0 \Rightarrow A = U \Lambda_A U^\dagger, B = U\Lambda_B U^\dagger.\\
	\sqrt{AB} = sqrt{U\Lambda_A U^\dagger U \Lambda_B U^\dagger} = \sqrt{U \Lambda_A \Lambda_B U^\dagger} = U\sqrt{\Lambda_A \Lambda_B}U^\dagger\\
	= U\sqrt\Lambda_A U^\dagger U \sqrt{\Lambda_B}U^\dagger = \sqrt A \sqrt B.
\end{gather}

\section{Solutions to the Dirac Equation}
\begin{equation}
	(i\slashed\partial - m) \psi = 0
\end{equation}
Let $\psi(x) = u(p)\cdot e^{ipx}$

Psi must also be a solution to the Klein Gordon equation.
\begin{equation}
	(\partial_\mu \partial^\mu + m^2)\psi = 0
\end{equation}
Starting:
\begin{gather}
	(i(-i\slashed p) - m)u(p) = 0 = (\slashed p - m)u(p)\\
	(-p^2 + m^2) u(p) \overset{!}{=} 0
\end{gather}
Which means that we require the On-Shell condition, so that $p^2 = m^2$.

Assume that we are in the rest frame, such that $p = (m,0)$. Thus
\begin{equation}
	\slashed p = \gamma^0 p_0 = m\gamma^0 = m
	\begin{pmatrix}
		0 & 0 & 1 & 0\\
		0 & 0 & 0 & 1\\
		1 & 0 & 0 & 0\\
		0 & 1 & 0 & 0
	\end{pmatrix}
\end{equation}
So (8) yields
\begin{equation}
	m
	\begin{pmatrix}
		-1 & 1\\
		1 & -1
	\end{pmatrix} u = 0, \; u =
	\begin{pmatrix}
		\xi \\ \xi
	\end{pmatrix},\; \xi = (1, 0)^T \text{ or } \xi = (0,1)^T
\end{equation}
Rotation: $u \mapsto S(\Lambda) u = 
\begin{pmatrix}
	e^{-i\varphi\sigma/2} & 0\\
	0 & e^{-i\varphi\sigma/2}
\end{pmatrix}
\begin{pmatrix}
	\xi \\ \xi
\end{pmatrix}
$\\
Boost: $u \mapsto S(\Lambda)u =
\begin{pmatrix}
	e^{-\eta\sigma/2} & 0\\
	0 & e^{+\eta\sigma/2}
\end{pmatrix}
\begin{pmatrix}
	\xi \\ \xi
\end{pmatrix}
$\\
$\cdots$
$$
u(p) =
\begin{pmatrix}
	\sqrt{p\sigma} \xi\\
	\sqrt{p\bar\sigma} \xi
\end{pmatrix}
$$
Claim: $\sqrt{(p\sigma)(p\bar\sigma)} = m$.

\textit{Proof:}\\
\begin{gather}
	(p\sigma)(p\bar\sigma) = (E - \vec p \vec\sigma)(E + \vec p\vec\sigma) = E^2 - (\vec p \vec\sigma)^2\\
	= E^2 - p^i \sigma^i p^j \sigma^j = E^2 - p^i p^j \cdot \frac{\sigma^i\sigma^j + \sigma^j\sigma^i}{2} = E^2 - p^i p^j = E^2 - \vec p ^2 = p^2 = m
\end{gather}
Verify that $u(p)$ satisfies the Dirac equation:
\begin{gather}
	\begin{pmatrix}
		-m & i\sigma\partial\\
		i\bar\sigma\partial & -m
	\end{pmatrix}
	\begin{pmatrix}
		\sqrt{p\sigma}\xi\\
		\sqrt{p\bar\sigma}\xi
	\end{pmatrix} e^{-ipx} =
	\begin{pmatrix}
		-m\sqrt{p\sigma}\xi + i\sigma(-ip)\sqrt{p\bar\sigma}\xi\\
		i\bar\sigma(-ip)\sqrt{p\sigma}\xi - m\sqrt{p\sigma}\xi
	\end{pmatrix}\\
	(-m\sqrt{p\sigma} + \sqrt{p\sigma}\sqrt{p\sigma}\sqrt{p\bar\sigma})\xi = 0
\end{gather}
So, it is zero and $u(p)$ fulfills the Dirac equation.

\section{Normalization}
\begin{equation}
	\bar\psi = \psi^\dagger \gamma^0,\;\;\bar u = u^\dagger \gamma^0
\end{equation}
So,
\begin{gather}
	(\xi^\dagger \sqrt{p\sigma}, \xi^\dagger \sqrt{p\bar\sigma})
	\begin{pmatrix}
		\sqrt{p\sigma}\xi \\ \sqrt{p\bar\sigma}\xi
	\end{pmatrix} = \xi^\dagger (p\sigma + p\bar\sigma)\xi  = 2E\xi^\dagger\xi
\end{gather}
\begin{gather}
	\bar u u = \xi^\dagger(\sqrt{p\sigma}\sqrt{p\bar\sigma} + \sqrt{p \bar\sigma}\sqrt{p\sigma})\xi = 2m\xi^\dagger \xi
\end{gather}

We have 2 linearly independent solutions $\psi(x) = u^s(p)e^{-ipx}, u^s(p)=
\begin{pmatrix}
	\sqrt{p\sigma}\xi^s \\ \sqrt{p \bar\sigma}\xi^s
\end{pmatrix},\;\; s = 1,2$.
and $\psi(x) = v^s(p)e^{ipx}, v^s(p) = 
\begin{pmatrix}
	\sqrt{p\sigma}\eta^s \\ -\sqrt{p \bar\sigma}\eta^s
\end{pmatrix}$

Orthogonality:
\begin{gather}
	\bar u^r(p) v^s(p) = (\xi^{r\dagger} \sqrt{p\sigma}, \xi^{r\dagger}\sqrt{p\bar\sigma})
	\begin{pmatrix}
		-\sqrt{p\bar\sigma}\eta^s \\ \sqrt{p\sigma}\eta^s
	\end{pmatrix} = \xi^{r\dagger}(-m + m)\eta^s = 0
\end{gather}
\begin{gather}
	u^{r\dagger}(\vec p)v^s(-\vec p) = (\xi^{r\dagger}\sqrt{p\sigma}, \xi^{r\dagger}\sqrt{p\bar\sigma})
	\begin{pmatrix}
		-\sqrt{p\bar\sigma}\eta^s\\
		\sqrt{p\sigma}\eta^s
	\end{pmatrix}\\
	E + \vec p \vec\sigma = p\bar\sigma,\;\; E - \vec p \vec\sigma = p\sigma
\end{gather}
$\bar u^r(p)u^s(p) = 2m\delta^{rs}$

\section{Spin sums}
When calculating cross sections, we will need:
\begin{gather}
	\sum_{s = 1,2} u^s(p)\bar u^s(p) = 
	\sum_{s=1,2}
	\begin{pmatrix}
		\sqrt{p\sigma}\xi^s \\ \sqrt{p\bar\sigma}\xi
	\end{pmatrix}
	(\xi^{s\dagger}\sqrt{p\sigma},\, \xi^{s\dagger}\sqrt{p\bar\sigma})\gamma^0\\
	\left[\sum_s \xi^s \xi^{s\dagger} = 1,\;\;\; \sum_s \ket s \bra s = 1\right]\\
	=
	\begin{pmatrix}
		p\sigma & \sqrt{p\sigma}\sqrt{p\bar\sigma}\\
		\sqrt{p\bar\sigma}\sqrt{p\sigma} & p\bar\sigma
	\end{pmatrix} \gamma^0 =
	\begin{pmatrix}
		m & p\sigma\\
		p\bar\sigma & m
	\end{pmatrix} = m + p^0\gamma^0 - p^i\gamma^i = m + \slashed p
\end{gather}
\begin{gather}
	\sum_s v^s(p)v^{-s}(p) = \cdots = \slashed p - m
\end{gather}

Addition: $m = \sqrt{(p\sigma)(p\bar\sigma)} = \sqrt{p\sigma}\sqrt{p\bar\sigma}$
\begin{gather}
	p\sigma = E - \vec p \vec\sigma,\;\; p\bar\sigma = E + \vec p \vec\sigma\\
	[p\sigma,p\bar\sigma] = 0.
\end{gather}

\section{$\gamma^5$}
\begin{gather}
	\gamma^5 = i\gamma^0\gamma^1\gamma^2\gamma^3\\
	(\gamma^5)^\dagger = -i \gamma^0\gamma^0\gamma^0 \gamma^0\gamma^1\gamma^0 \gamma^0\gamma^2\gamma^0 \gamma^0\gamma^3\gamma^0 = -i\gamma^1\gamma^2\gamma^3\gamma^0 = i\gamma^0\gamma^1\gamma^2\gamma^3\\
	(\gamma^5)^2 = 1, \;\; \gamma^5 =
	\begin{pmatrix}
		- 1 & 0\\ 0 & 1
	\end{pmatrix}\\
	\{\gamma^5, \gamma^\mu\} = 0,\; \gamma^5\gamma^\mu = i\gamma^0\gamma^1\gamma^2\gamma^3\gamma^\mu = -\gamma^\mu\gamma^5
\end{gather}

\end{document}

