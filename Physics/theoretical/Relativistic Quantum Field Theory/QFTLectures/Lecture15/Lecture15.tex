\documentclass[]{scrartcl}

\usepackage{\string~"/LaTeX/StylePackage"}
\usepackage{iftex}

\title{QFT - Lecture 15}
\author{}
\date{14.10.2024}


\begin{document}

I have been compiled with\dots{\ttfamily
\ifluatex
    luatex
\else\ifxetex
    xetex
\else\ifpdftex
    pdftex
\else
    NO IDEA!
\fi\fi\fi}

\maketitle
\newpage
\tableofcontents
\newpage

\section{Correlation functions} 

\begin{equation}
\bra\Omega T\phi(x)\phi(y)\ket\Omega = \bra0 T\phi(x)\phi(y)e^{-i\int\text dt H_I(t)}\ket 0
\end{equation}
with $\phi(x)$, $\phi(y)$ being in the Heisenberg picture on the left and in the Interaction picture on the right.
$$
H_I(t) = \int\frac{\lambda}{4!}\phi^4 \text d^3x
$$
with $\phi^4$ in the interaction picture.

\section{Wick's Theorem}

$$
T \phi_1\cdots\phi_n = N(\phi_1\cdots\phi_n + \text{all possible contractions})
$$
For example:
\begin{gather}
	\bra0 T\phi_1\phi_2\phi_3\phi_4 \ket0 = \wick{
	\c1\phi_1\c1\phi_2\c1\phi_3\c1\phi_4 + \c1\phi_1\c2\phi_2\c1\phi_3\c2\phi_4 + \c2\phi_1\c1\phi_2\c1\phi_3\c2\phi_4
	}
\end{gather}
Which can be represented as the Feynman Diagrams
\begin{equation}
\feynmandiagram[vertical=a to b]{
	{a [particle = \(x_1\)]} -- {b [particle = \(x_2\)]}
{c [particle = \(x_3\)]} -- {d [particle = \(x_4)\]},
};	
\end{equation}

\begin{tikzpicture}
\begin{feynman}
\diagram [vertical'=a to b] {
i1 [particle=\(e^{-}\)]
-- [fermion] a
-- [draw=none] f1 [particle=\(e^{+}\)],
a -- [photon, edge label'=\(p\)] b,
i2 [particle=\(e^{+}\)]
-- [anti fermion] b
-- [draw=none] f2 [particle=\(e^{-}\)],
};
\diagram* {
(a) -- [fermion] (f2),
(b) -- [anti fermion] (f1),
};
\end{feynman}
\end{tikzpicture}


$Eq1:$ \\ lowest order: Num: $\bra0 T\phi(x)\phi(y)\ket0 = D_F(x-y) =$ Fig2\\
first order: Num: $\bra0 T\phi(x)\phi(y)\ket0 (-i\lambda/4!)\int \text d^4z \phi(z)^4\bra0$\\
We now have 6 fields and must fiind all contractions. We can contract $\phi(x)$ and $\phi(y)$ with $\phi(z)$, which leads to $4\cdot3 = 12$ contractions. We can also contract $\phi(x)$ with $\phi(y)$ This leads to $3$ contractions.\\
Which leads to
\begin{equation}
	\frac{-i\lambda}{4!}\int \text d^4z 3D_F(x-y)D_F(z-z)D_F(z-z) + 12 D_F(x-z)D_F(y-z)D_F(z-z)
\end{equation}
With the Feynman Propagator
\begin{equation}
	D_F(x-y) = \frac{\text d^4p}{(2\pi)^4}\frac{i}{p^2 + m^2 + i\epsilon} e^{-ip(x-y)}
\end{equation}
We can see that this integral becomes infinite for $x = y$, and integrating over all space again afterwards will definitely lead to infinity. We will ignore this for now, the divergences will disappear later.

For the 3 contractions we have the following Feynman Diagrams: Fig3\\\\
For the 12 contractions we have the following Feynman Diagrams: Fig4\\\\
Now considering a more complicated situation, with the third order:
\begin{gather}
	\bra0 T\phi(x)\phi(y) \frac{1}{3!}\left(\frac{-i\lambda}{4!}\right)^3 \int\text d^4z \phi^4 \int \text d^4w \phi^4 \int \text d^4u \phi(u)
\end{gather}
We will consider one particular contraction, that being
%\begin{equation}
%	\wick{
%	\c1\phi(x)\c2\phi(y)\c1\phi(z)\c3\phi(z)\c3\phi(z)\c4\phi(z)\c5\phi(w)\c5\phi(w)\c6\phi(w)\c2\phi(w)\c5\phi(u)\c4\phi(u)\c1\phi(u)\c1\phi(u)	}
%\end{equation}
With the Feynmann Diagram Fig5

How many contractions give the same expression? We get $3!$ from interchanging verteces, $4\cdot3$ from the placement of contractions in the $z$ vertex, we then get $4\cdot3\cdot2$ placements of the $w$ vertex, and $4\cdot3$ from the $u$ vertex. We have overcounted by a factor of $2$, as we can interchange the contraction between the $w$ and $u$ fields, so we need to divide by 2.\\\\
we ignore $\frac{1}{3!}3!$ and we ignore $\frac{1}{4!}\cdot4\cdot3$, $\frac{1}{4!}4!$. We then divide by the so called symmetry factor $s$, in our case $s=2\cdot2\cdot2$. The first 2 comes from the line starting and ending at the $z$ vertex, the 2nd from the line starting and ending at the $u$ vertex, and the last because we can interchange the $wu$ lines.

Another possibility of getting a symmetry factor is if two vertices are equivalent. Possible symmetry factors are as following
\begin{itemize}
	\item line starting and ending at the same vertex
	\item equivalent lines
	\item equivalent vertices
\end{itemize}
These Symmetry factors are the reason why we should divide by the symmetry factor instead of leaving $\frac{1}{3!}$ and $\frac{1}{4!}$ in the third order term.
\subsubsection{Summary:}
\begin{gather}
	\bra\Omega T\phi(x)\phi(y)\ket\Omega = \frac{\text{Num}}{\text{Den}}\\
	\text{Num: } \sum \text{all possible diagrams with two external points}
\end{gather}
Feynman rules: Figs7\\\\

Fig8\\\\

Momentum Space Feynman Rules\\\\

Momentum Space Feynman Rules, correlation functions to Fourier space.
\begin{equation}
	\bra\Omega T\phi(x)\phi(y)\ket\Omega \mapsto \int \text d^4x e^{ipx} \int \text d^4y e^{ipy}\bra\Omega T\phi(x)\phi(y) \ket\Omega
\end{equation}
Omit exponent factors of external points, and the associated $\int\frac{\text d^4p}{(2\pi)^4}$ Integrals.
\begin{equation}
	f(x) \mapsto \hat f(p),\;\;\; \hat f(p) \mapsto f(x),\;\;\; f(x) = \int\frac{\text d^4p}{(2\pi)^4}\hat f(p) e^{-ipx}
\end{equation}
That way we can get $\hat f(p)$ by itself.

\section{Exponentiation and cancellation}
Consider Fig9

Then we have a product of two delta functions, which is nonsense\\
Let $V_i\in\{\text{The set of all different diagrams disconnected from external points}\}$. 
$$
A= \text{(Value of connected piece)}\cdot\prod_{i}\frac{1}{n_i!}V_i^{n_i},\;\;\; n_i = \text{number of $V_i$.}
$$
$$
\text{Num=} \sum_{\text{all connected}}\text{(value connected)}\cdot\underbrace{\prod_i\sum_{n_i}\frac{1}{n_1!}V_i^{n_i}}_{\exp(\sum_i V_i)}
$$
$$
\text{Den=} \exp(\sum_i V_i)
$$
and therefore what we would like to have is
$$
\frac{\text{Num}}{\text{Den}} = \sum\text{all connected}
$$

\end{document}

