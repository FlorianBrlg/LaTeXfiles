\documentclass[]{scrartcl}

\usepackage{\string~"/LaTeX/StylePackage"}

\title{Lecture 2}
\author{}
\date{\today}


\begin{document}

\maketitle
\newpage
\tableofcontents
\newpage

 
For a solid crystal, when heated such that it becomes a liquid, you need to add a "latent heat", and the Energy is used to break the crystal bonds.

\section{basic concepts}

\subsection{heat capacity and latent heat}

The heat capacity is defined as $c = \frac{Q}{\Delta T}$, so it's the Heat $Q$ needed to raise the Temperature by $\Delta T$. We also define the specific heat capacity $c_v$ such that $c = c_v\cdot m$.

To transform a solid to a liquid, we need the latent heat $L_m$, so to raise the temperature from below the solid temperature to the liquid temperature, the required heat is $Q = c\Delta T + L_m$. Which we then also define as $Q = (c_v\Delta T + \ell_m)m$ where $\ell_m$ is the specific latent heat.

\subsection{Thermal Radiation}
\subsubsection{Reflection, Absorption and Transfer}
Every material Reflects, absorbs and transfers the incoming Thermal Radiation. We then have
\begin{equation}
	J_Q = J_{Q,R} + J_{Q, a} + J_{Q, T}
\end{equation}
Where the Reflectivity, Absorptivity and Transmissivity are defined as
\begin{gather}
	\rho = \frac{J_{Q,R}}{J_Q}\\
	\alpha = \frac{J_{Q,a}}{J_Q}\\
	\tau = \frac{J_{Q,T}}{J_Q}
\end{gather}

We also have the Emissivity
\begin{equation}
	\epsilon = \alpha
\end{equation}
Where for example for Black bodies $J_Q = J_{Q,a}$ and $\epsilon = \alpha = 1$. We also know, that for a Black body
\begin{equation}
	J_Q = \sigma T^4
\end{equation}
and for real bodies
\begin{equation}
	J_Q = \epsilon\sigma T^4
\end{equation}

In the Summer and Winter month, we want to keep our House cold, or warm. When we have a Temperature $T_1$ and $T_2$, and a wall between it with thickness $\Delta x$ then there is a Flux $J_Q$ through it. With a temperature difference $\Delta T = T_2 - T-1$. We define $T_1$ be inside and $T_2$ be outside.

If we double the Temperature difference, then the Flux is also doubled.

If we double the width of the wall, then the Flux is halfed. We therefore define
\begin{equation}
	J_Q = \lambda \frac{\Delta T}{\Delta x} = \underbrace{\pm}_{?}\lambda\nabla T
\end{equation}
where $\lambda$ is the thermal conductivity.

\subsection{Heat conduction experiment}

The temperature curve of just the hot block is linear when looking at the temperature on a logarithmic scale, thus:
\begin{equation}
	\log{(T-T_{Room})} \propto -t
\end{equation}
Thus
\begin{equation}
	T - T_{Room} = \Delta T\propto e^{-t/c}
\end{equation}
And for the temperature difference of the hot and cold block we have
\begin{equation}
	T_h - T_c \propto e^{-t/\tau_1}
\end{equation}
We also see that both materials had different Temperature changes.
\begin{equation}
	\Delta T_B = 30°K
\end{equation}
\begin{equation}
	\Delta T_Y = 60°K
\end{equation}
Blue = hot, Yellow = cold

Then
\begin{equation}
	\frac{\Delta T_h}{\Delta T_c} = \frac{Q/c_h}{Q/c_c} = \frac{c_h}{c_c} = \frac{1}{2}
\end{equation}

\subsection{Thermal Conduction, boundary condition}

\begin{gather}
	J_Q = -\lambda \nabla T\\
	Q = c\Delta T
\end{gather}
use Energy conservation in the form of the continuity equation.
\begin{gather}
	\pdv{Q}{t} + V\nabla J_Q = 0\\
	\pdv{Q}{t} - \lambda V \nabla^2 T = 0\\
	c\pdv{T}{t} - \lambda V\nabla^2 T = 0\\
	\pdv{T}{t} - \frac{\lambda V}{c_v m} \nabla^2 T = 0\\
	\pdv{T}{t} - \frac{\lambda}{c_v \rho}\nabla^2 T = 0
\end{gather}




\end{document}

