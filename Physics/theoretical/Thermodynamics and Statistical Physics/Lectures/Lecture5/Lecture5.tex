\documentclass[]{scrartcl}

\usepackage{\string~"/LaTeX/StylePackage"}

\title{Thermodynamics and Statistical Mechanics}
\author{}
\date{3.9.2024}


\begin{document}

\maketitle
\newpage
\tableofcontents
\newpage


\subsection{Ideal Gas}

\begin{itemize}
	\item Point particles that don't fill up space
	\item Non interacting particles
	\item Newtonian physics, $m\frac{\text d v}{\text d t} = F$
	\item The particles have a Kinetic Energy $T = \frac{1}{2}mv^2$

\end{itemize}

We have a Box with length $\ell$ and a particle interacting with the surface $A$ with its $v_x$ velocity.

The time between colissions with the surface is $\Delta t = \frac{2 \ell}{v_x}.$ We calculate the Pressure with
\begin{equation}
	P = \frac{F}{A} = \frac{m\Delta v}{A \Delta t} = \frac{m2v_x^2}{A2\ell} = \frac{mv_x^2}{V}
\end{equation}
And then
\begin{equation}
	PV = mv_x^2 = kT
\end{equation}
from this we gather
\begin{itemize}
	\item Equipartition principle: every squared degree of freedom has $\frac{1}{2}kT$ Energy.
\end{itemize}
And the total Energy for the particle is
\begin{equation}
	E = \sum_i \frac{1}{2}m<v_i^2> = \frac{3}{2}m<v^2> = \frac{3}{2}kT
\end{equation}



\section{Molecular Dynamics}
Lennard Jones Potential
\begin{equation}
	U(r) = 4\epsilon \left(\left(\frac{\sigma}{r}\right)^{12}-\left(\frac{\sigma}{r}\right)^6\right)
\end{equation}
For example
\begin{itemize}
	\item Argon
		\begin{itemize}
			\item $\sigma = 0.34$ nm
			\item $\epsilo/k_B = 120$K
			\item $m=40u$
		\end{itemize}
	\item Methane
		\begin{itemize}
			\item $\sigma=0.38$
			\item $\epsilon/k_B = 148$K
			\item $m = 16u$
		\end{itemize}
\end{itemize}

 



\end{document}

