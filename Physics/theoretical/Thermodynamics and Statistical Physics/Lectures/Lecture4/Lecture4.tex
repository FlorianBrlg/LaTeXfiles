\documentclass[]{scrartcl}

\usepackage{\string~"/LaTeX/StylePackage"}

\title{QFT - Lecture 4}
\author{}
\date{28.8.2024}


\begin{document}

\maketitle
\newpage
\tableofcontents
\newpage

\section{Microstates and Macrostates}
\begin{itemize}
	\item what is the most likely outcome of tossing 3 coins?
	\item Microstates: state of all coins
		\begin{itemize}
			\item heads: $s_i = 1$, tails $s_i = 0$.
			\item all microstates are equally likely
		\end{itemize}
	\item Macrostate: collective property
		\begin{itemize}
			\item sum of states: $n = \sum_i s_i = \{0,1,2,3\}$
		\end{itemize}
	\item wWhich is the most likely macrostate?
		\begin{itemize}
			\item $8 = 2^3$ possible microstates
			\item Probabilities:
				\begin{itemize}
					\item n = 0, P = 1/8
					\item n = 1, P = 3/8...
				\end{itemize}
		\end{itemize}
\end{itemize}

Some Two-State models:
\begin{itemize}
	\item Paramagnets
	\item two-sided box
	\item random walk
\end{itemize}
System: $N$ spins, particles, steps, coins
\begin{itemize}
	\item Isolated
	\item Independent (no interaction between spins/particles, no correlation between successive steps)
	\item Distinguishable (the order matters)
	\item Equal probability of states $s_i = \pm1$
\end{itemize}

Macroscopic explanation of diffusion:\\
Net transport of energy of particles until thermodynamics equilibrium is reached

\begin{itemize}
	\item $J = -D\nabla c$ Matter flux is proportional to gradient of concentration
	\item $Q = -\lambda\nabla T$ Heat flux is proportional to gradient of temperature
	\item what are the equilibrium conditions?
\end{itemize}

Microscopic explanation of diffuction
\begin{itemize}
	\item[] Net transport of energy or particles
		\begin{itemize}
			\item through random thermal motion and particle collisions
			\item until the most likely states are reached
		\end{itemize}
	\item At any $T>0K$, particles are in thermal motion
	\item Collisions between particles is a zigzag
\end{itemize}

Microscopic models of diffusion
\begin{enumerate}
	\item Random walk
	\item Algorithmic
	\item Molecular Dynamics
\end{enumerate}
\begin{itemize}
	\item Reversible laws of motion
	\item Irreversible development: arrow of time
	\item Measure average property: distribution in box (left/right)
\end{itemize}

\section{Einstein Crystal}

There are $N$ independent, distinguishable quantum harmonic oscillators.\\
They have discrete energy states $E_n = (n + \frac{1}{2})\hbar\omega$\\
The microstates of this crystal are $\{n_i\}; \{n_1, n_2, n_3, \cdots, n_N\}$\\
Macrostate $U_N = \sum_{i \in \{n_i\}} \E_i = \sum_i n_i\hbar\omega + \frac{1}{2}\hbar\omega$
\begin{equation}
	q = \frac{U_N - \frac{1}{2}\hbar\omega}{\hbar\omega} = \sum_i n_i
\end{equation}
The macrostates are defined by $(N,q)$.

Example: 3 oscillators.
\begin{equation}
	q = 2
\end{equation}
we have 6 microstates for this macrostate. We will map this problem onto a two state model. We say that our oscillators are 3 boxes, where the walls are 1s and the balls are 0s. Then for example $\{0,0,2\} = 1100$. This fully describes our system with a $4$-bit number. Then we calculate
\begin{gather}
	N'=4 = N - 1 + q = \text{ walls + balls}\\
	n = q\\
	\Rightarrow \Omega(N, q) = \frac{(N-1+q)!}{q!(N-1)!}
\end{gather}

\end{document}

