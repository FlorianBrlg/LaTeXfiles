\documentclass[]{scrartcl}

\usepackage{\string~"/LaTeX/StylePackage"}
\title{Homework Sheet 02}
\author{Florian, Elias}
\date{\today}


\begin{document}

\maketitle
\newpage
\tableofcontents
\newpage

\section{Canonical Transformations and Classical Trajectories}
\subsection{}
We require that $(q_i, p_i)$ is a valid trajectory, so we require that $\frac{\text dq_i}{\text dt} = \pdv{H}{p_i},\;\frac{\text dp_i}{\text dt} = -\pdv{H}{q_i}$. We want to see that
$$
\frac{\text d \bar q_i}{\text dt},\;\;\;\;\frac{\text d \bar p_i}{\text dt}
$$
Are valid trajectories.

We will use the mappings
\begin{equation}
q_i \mapsto \bar q_i = q_i + \epsilon\pdv{g}{p_i},\;\;\;\; p_i \mapsto \bar p_i = p_i - \epsilon\pdv{g}{q_i}
\end{equation}
and that
$$
\{g,H\} = 0 = \pdv{g}{q_i}\pdv{H}{p_i} - \pdv{g}{p_i}\pdv{H}{q_i} = \pdv{g}{q_i}\pdv{q_i}{t} + \pdv{g}{p_i}\pdv{p_i}{t} = \pdv{g}{t}
$$
%We will split the transformed Hamiltonian into two parts:
%\begin{gather}
%	H\mapsto H'(\bar p, \bar q) = H(p, q) + \delta H
%\end{gather}
%and calculate $\delta H$.
%\begin{gather}
%	\delta H = \pdv{H}{q}\delta q + \pdv{H}{p}\delta p\\
%	\text{using the definitions of $\delta q,\delta p$ from Eq.1}\nonumber\\
%	\delta H = \epsilon\left(\pdv{H}{q}\pdv{g}{p} - \pdv{H}{p}\pdv{g}{q}\right) = 0
%\end{gather}
We calculate explicitly:

\begin{gather}
	\frac{\text d \bar q}{\text dt} = \frac{\text dq}{\text dt} + \epsilon \frac{\partial^2 g}{\partial t\partial p} = \frac{\text dq}{\text dt}\\
	\frac{\text d \bar p}{\text dt} = \frac{\text dp}{dt} - \epsilon \frac{\partial^2 g}{\partial t\partial q} = \frac{\text dp}{\text dt}
\end{gather}
So, as we know that $(q,p)$ describes a valid trajectory, and the time derivatives of $(\bar q,\bar p)$ are equivalent, they must also describe a valid trajectory.



\subsection{}

we have a transformation
$$
x_k \mapsto x'_k = x_k + \delta
$$
We will take a look at the generating function. We know that for transformation of the coordinate $\delta = \delta\pdv{g}{p_k}$. We also know that we do not transform our conjugate momentum, so $0 = \delta\pdv{g}{q_k}$. Therefore we know that
$$
g = p
$$
WE want to check if this transformation gives us a valid trajectory if $(x_k, p_k)$ is a valid trajectory, hence we will use the method that we have used in the previous task. We will calculate the Poisson bracket:
\begin{gather*}
    \{p,H\} = \underbrace{\pdv{p}{q_i}\pdv{H}{p_i}}_{=0} - \pdv{p}{p_i}\pdv{H}{q_i}\\
    = - \pdv{p}{p_i}\pdv{H}{q_i} = - e_i \pdv{H}{q_i}\\
    = e_i \dot p_i = \dot p\\
    \{p,H\} = \pdv{p}{q_i}\pdv{H}{p_i} - \pdv{p}{p_i}\pdv{H}{q_i} = \pdv{p}{q_i}\pdv{q_i}{t} + \pdv{p}{p_i}\pdv{p_i}{t}\\
    \pdv{p}{t} + \pdv{p}{t} = 2\dot p\\
    \Rightarrow \{p,H\} = \dot p = 2\dot p = 0
\end{gather*}
Now that we know that the Poisson bracket is 0 we can apply what we have found out from the previous task, and infer that $(x'(t), p(t))$ is a valid trajectory, seeing as $(x(t), p(t))$ is a valid trajectory, and the Poisson bracket $\{g,H\} =0$.



\section{Canonical Transformation in Quantum Mechanics}

\subsection{}
We apply the operator $U(\xi) = e^{-i\xi g/\hbar}$ on the wave function
\begin{gather*}
e^{-i\xi g/\hbar}\psi(q_i,t) = \sum_n e^{-i\xi g/\hbar}c_n(t)\psi_n(q_i)\\
=  \sum_n c_n(t) e^{-i\xi g_n/\hbar}\psi_n(q_i)\\
= \sum_n e^{-i\xi g_n/\hbar}c_n(t) \psi_n(q_i) = \sum_n \bar c_n(t) \psi_n(q_i)
\end{gather*}
we from this we infer $\bar c_n = e^{-i\xi g_n/\hbar}c_n$.

\subsection{}
We express our Quantum Operator with $g = L_z$, such that $U(\xi) = \exp(-i\xi L_z/\hbar)$

We know the form of $L_z$, but more importantly we know the Eigenvalues of $L_z$. We will take $\psi_n$ to be Eigenfunctions of $L_z$, that means that $\psi_n(\rho,\phi,\theta) = f(\rho,\theta)e^{-in\phi}$. Then
\begin{gather*}
	e^{-i\xi L_z/\hbar} \psi = e^{-i\xi L_z/\hbar}\sum_n c_n f(\rho,\theta)e^{-in\phi}\\
	= \sum_n c_n e^{-i\xi n}f(\rho,\theta)e^{-in\phi}\\
	= \sum_n c_n f(\rho,\theta)e^{-in(\phi + \xi)}
\end{gather*}
From this we can see that $g = L_z$ generates a shift in $\phi$, which we define as the rotation around the $z$ axis, so $L_z$ generates a rotation around the $z$ axis.

\subsection{}

\section{Gauge Invariance in Classical Electrodynamics}

\subsection{}
We will show this
\begin{gather*}
	\nabla \times A \mapsto \nabla \times (A + \nabla \lambda)\\
= \nabla \times A + \underbrace{\nabla\times\nabla\lambda}_{=0} = \nabla\times A
\end{gather*}

\begin{gather*}
	E = -\nabla U - \pdv{A}{t} \mapsto -\nabla\left(U - \pdv{\lambda}{t}\right) - \pdv{A + \nabla\lambda}{t}\\
	= -\nabla U + \nabla\pdv{\lambda}{t} - \pdv{A}{t} - \nabla\pdv{\lambda}{t}\\
	= -\nabla U - \pdv{A}{t}
\end{gather*}
Therefore, the $E$ and $B$ fields stay invariant.

\subsection{}

for the $B$ field
\begin{gather}
	\nabla B = \nabla (\nabla\times A) = 0
\end{gather}
for the $E$ field
\begin{gather}
	\nabla \times E = -\nabla\times (\nabla U) - \nabla \times \pdv{A}{t}\\
	= -\pdv{}{t}\nabla \times A = -\pdv{B}{t}
\end{gather}

\subsection{}

We plug in the expressions for our fields from the potentials:

\begin{gather*}
\nabla\left(-\nabla U - \pdv{A}{t}\right) =\frac{\rho}{\epsilon_0}\\
-\Delta U - \pdv{\nabla A}{t} = \frac{\rho}{\epsilon_0}\\
-\Delta U + \mu_0\epsilon_0\pdv{^2U}{t^2} = \frac{\rho}{\epsilon_0}
\end{gather*}
and
\begin{gather*}
	\nabla\times(\nabla\times A) = \mu_0 j + \mu_0\epsilon_0\pdv{-\nabla U - \pdv{A}{t}}{t}\\
	\nabla\times(\nabla\times A) = \mu_0 j + \mu_0\epsilon_0\left(-\nabla\pdv{U}{t} - \pdv{^2A}{t^2}\right)\\
\nabla(\nabla A) - \nabla^2 A = \mu_0j - \mu_0\epsilon_0\pdv{^2A}{t^2} -\mu_0\epsilon_0\nabla \pdv{U}{t}\\
	-\nabla\left(\mu_0\epsilon_0 \pdv{U}{t}\right) - \nabla^2 A = \mu_0j - \mu_0\epsilon_0\pdv{^2A}{t^2} -\mu_0\epsilon_0 \nabla\pdv{U}{t}\\
	-\Delta A + \mu_0\epsilon_0\pdv{^2A}{t^2} = \mu_0 j
\end{gather*}

Both of these equations could be made much prettier with the co and contravariant derivative.

\subsection{}

The fields stay invariant, so I will only consider the changes in the Potentials themselves:
\begin{gather*}
	-\rho U + jA \mapsto -\rho U + \rho \pdv{\lambda}{t} + jA + j \nabla\lambda
\end{gather*}
Charge is conserved, this means that
\begin{gather*}
	\partial_t \rho + \nabla j = 0\\
	\int \text d^4x \partial_t \rho+ \nabla j = 0
\end{gather*}
We will consider the Spacetime Integral over the additional terms of the Lagrangian Density.
\begin{gather*}
	\delta S = \int \text d^4x \rho\pdv{\lambda}{t} + j\nabla\lambda\\
	\text{integration by parts, and the surface terms of $\lambda$ are ignored.}\\
	= -\lambda \int \text d^4x \partial_t \rho + \nabla j = 0
\end{gather*}
hence, there is no change in the Action, so it stays Invariant.
\end{document}

