\documentclass[]{scrartcl}

\usepackage{\string~"/LaTeX/StylePackage"}

\title{Theoretical Particle Phyics - Sheet 01}
\author{}
\date{\today}


\begin{document}

\maketitle
\newpage
\tableofcontents
\newpage

\section{Natural Units} 

$$
\hbar = c = k_B = 1
$$
which is
$$
J\cdot s = \frac{m}{s} = \frac{J}{\cdot K} = 1
$$
which we can show as:
$$
J = s^{-1},\;\;\; m = s,\;\;\; J = K
$$

\subsection{a)}
\begin{gather}
	1K = 1.380\cdot10^{-23}J = 8.61\cdot10^{-14}GeV
\end{gather}
Mass and Energy dimension of one

\subsection{b)}

\begin{gather}
	1g = 4.494\cdot10^{15}J = 2.8\cdot10^{25}GeV
\end{gather}
Mass and Energy dimension of one

\subsection{c)}

\begin{gather}
	1cm = 1.509\cdot10^{34}J^{-1} = 
\end{gather}
The conversions aren't working at all I give up\\
Mass and Energy Dimension of minus one

\subsection{d)}

Mass and energy dimension of minus two 

\subsection{e)}
Mass and energy dimension of one

\section{The Lorentz Group Part1}

\subsection{a)}
we require that
$$
g_{\mu\nu} x^\mu x^\nu = g_{\mu\nu} x'^\mu x'^\nu
$$
With the transformation $x'^\mu = \Lambda^\mu_\nu x^\nu$. This leads to
\begin{gather*}
	g_{\mu\nu}x^\mu x^\nu = g_{\rho\sigma} x'^\rho x'^\sigma\\
	g_{\mu\nu}x^\mu x^\nu = g_{\rho\sigma} \Lambda^\rho_\mu x^\mu \Lambda^\sigma_\nu x^\nu\\
	\Rightarrow g_{\mu\nu} = g_{\rho\sigma} \Lambda^\rho_\mu \Lambda^\sigma_\nu
\end{gather*}

\subsection{b)}
First off we produce the required relation:
\begin{gather}
	g_{\mu\nu} = g_{\rho\sigma}\Lambda^\rho_\mu\Lambda^\sigma_\nu\nonumber\\
	1 = g_{\rho\sigma}\Lambda^\rho_\mu\Lambda^\sigma_\nu g^{\mu\nu}\nonumber\\
	(\Lambda^{-1})^\nu_\sigma = g_{\rho\sigma}\Lambda^\rho_\mu g^{\mu\nu}\nonumber\\
	\text{rename indeces to get it into the proper form}\nonumber\\
	(\Lambda^{-1})^\nu_\mu = g^{\nu\sigma}\Lambda^\rho_\sigma g_{\rho\mu}\label{eq:Lambda}
\end{gather}

We want to see that the covariant coordinate $x_\mu = g_{\mu\nu}x^\nu$ transform under Lorentz transformation as
$$
x_\mu \mapsto x'_\mu = x_\nu (\Lambda^{-1})^\nu_\mu
$$
So we will insert the previous expression \ref{eq:Lambda}
\begin{gather}
	x'_\mu = x_\nu g^{\nu\sigma}\Lambda^\rho_\sigma g_{\rho\mu}\\
	= g_{\nu\alpha}x^\alpha g^{\nu\sigma}\Lambda^\rho_\sigma g_{\rho\mu}\\
	= x^\alpha \delta^\sigma_\alpha \Lambda^\rho_\sigma g_{\rho\mu} = x^\sigma \Lambda^\rho_\sigma g_{\rho\mu}\\
	= x^\sigma \Lambda^{\rho\nu}g_{\nu\sigma}g_{\rho\mu} = x_\nu\Lambda^\nu_\mu\\
	= \Lambda^\nu_\mu x_\nu
\end{gather}
And therefore, that is how the covariant coordinate transforms under Lorentz Transformations.

\subsection{c)}

We will consider a covariant derivate of the new variable $x'$
\begin{gather}
	\partial_\mu\mapsto \pdv{}{x'^\mu}\\
	\text{which we can expand with the chain rule to get the derivative of the old coordinate}\\
	=\pdv{x^\nu}{x'^\mu}\pdv{}{x^\nu}\\
	\text{and using the previous definition of the Lorentz transformation}\\
	=\left(\Lambda^{-1}\right)^\nu_\mu \partial_\nu
\end{gather}
Which is what we wanted to show.

\subsection{d)}

We consider to contravariant derivative, but use the metric tensor to express it as the covariant derivative:
\begin{gather}
	\partial^\mu = g^{\mu\sigma}\partial_\sigma \mapsto g^{\mu\sigma} \left(\Lambda^{-1}\right)^\nu_\sigma \partial_\nu\\
	= g^{\mu\sigma}g^{\nu\rho}\Lambda^\alpha_\rho g_{\alpha\sigma}\partial_\nu = \delta^\mu_\alpha \Lambda^\alpha_\rho \partial^\rho\\
	= \Lambda^\mu_\rho\partial^\mu
\end{gather}

\subsection{e)}
From Eq5:
\begin{gather}
	g = A^T gA\\
	\Rightarrow \det(g) = \det(A^T g A)\\
	\Rightarrow \det(g) = \det(A^T)\det(A)\det(g)\\
	\det(A^T) = \det(A)\nonumber\\
	\Rightarrow 1 = \det(A)^2\\
	\Rightarrow 1 = |\det(A)|
\end{gather}
We will inspect the first element closer
\begin{gather}
	g_{00} = \left(\Lambda^T g \Lambda\right)_{00}\\
	g_{00} = g_{\alpha\beta}\Lambda^0_\alpha\Lambda^0_\beta\\
	g_{00} = \left(\Lambda^0_0\right)^2 -\left(\Lambda^0_i\right)^2\\
	1 \leq \left(\Lambda^0_0\right)^2
\end{gather}

\subsection{f)}

wdym?

\subsection{g)}

\begin{gather}
	\text{consider } A(t) = \exp(tL)\text{ then, though Jacobi's formula:}\\
	\frac{\text d}{\text dt}\det(A) = \det(A) \text{tr}\left[A^{-1} \frac{\text d}{\text dt}A\right]\\
\Rightarrow \frac{\text d}{\text dt}\det(A) = \det(A)\text{tr}(tL)\\
\text{This is a DiffEq and yields the result}\\
\det(A) = e^{\text{tr}(tL)}
\end{gather}
setting $t=1$ then gives us the wanted result. We can show that tr$(L) = 0$ by the fact that
\begin{gather}
	1 = e^{\text{tr}(L)}\\
	\Rightarrow \ln(1) = \ln\left(e^{\text{tr}(L)}\right)\\
	\Rightarrow 0 = \text{tr}(L)
\end{gather}


\subsection{h)}

we have the defining property
\begin{equation}
	g = \Lambda^T g \Lambda
\end{equation}
with $\Lambda = \exp(L)$ becomes
\begin{gather}
	g = \exp(L)^T g \exp(L)\\
	g\exp(-L) = \exp(L)^T g\\
	\exp(-gL) = \exp((gL)^T)\\
	-gL = (gL)^T
\end{gather}
keeping in mind that $L$ is traceless, the most general matrix will look like
$$
\begin{pmatrix}
	0 & a & b & c\\
	a & 0 & d & e\\
	b & -d & 0 & f\\
	c & -e & -f & 0
\end{pmatrix}
$$
the form of $d,e,f$ comes from the fact that $g$ transforms the lower right $3\times3$ portion of the matrix into the negative, while leaving $a,b,c$ as is, meaning that if we take the transpose we willl have to negate all the elements of the $3\times3$ matrix again, so that the equality holds.

There are $6$ free variable, so a basis can be constructed out of $6$ matrices.

\subsection{i)}

\begin{gather*}
	S_3^2 = 
	\begin{pmatrix}
		0 & 0 & 0 & 0\\
		0 & 0 & -1& 0\\
		0 & 1 & 0 & 0\\ 
		0 & 0 & 0 & 0
	\end{pmatrix}\cdot
	\begin{pmatrix}
		0 & 0 & 0 & 0\\
		0 & 0 & -1& 0\\
		0 & 1 & 0 & 0\\ 
		0 & 0 & 0 & 0
	\end{pmatrix} =
	\begin{pmatrix}
		0 & 0 & 0 & 0\\
		0 & -1& 0 & 0\\
		0 & 0 & -1& 0\\ 
		0 & 0 & 0 & 0
	\end{pmatrix}\\
	S_3^3 = \begin{pmatrix}
		0 & 0 & 0 & 0\\
		0 & -1& 0 & 0\\
		0 & 0 & -1& 0\\ 
		0 & 0 & 0 & 0
	\end{pmatrix}\cdot
	\begin{pmatrix}
		0 & 0 & 0 & 0\\
		0 & 0 & -1& 0\\
		0 & 1 & 0 & 0\\ 
		0 & 0 & 0 & 0
	\end{pmatrix} = 
	\begin{pmatrix}
		0 & 0 & 0 & 0\\
		0 & 0 & 1& 0\\
		0 & -1 & 0 & 0\\ 
		0 & 0 & 0 & 0
	\end{pmatrix} = -S_3\\
	K_1^2 = 
	\begin{pmatrix}
		0 & 1 & 0 & 0\\
		1 & 0 & 0 & 0\\
		0 & 0 & 0 & 0\\ 
		0 & 0 & 0 & 0
	\end{pmatrix} \cdot 
	\begin{pmatrix}
		0 & 1 & 0 & 0\\
		1 & 0 & 0 & 0\\
		0 & 0 & 0 & 0\\ 
		0 & 0 & 0 & 0
	\end{pmatrix} =
	\begin{pmatrix}
		1 & 0 & 0 & 0\\
		0 & 1 & 0 & 0\\
		0 & 0 & 0 & 0\\ 
		0 & 0 & 0 & 0
	\end{pmatrix}\\
	K_1^3 = \begin{pmatrix}
		0 & 1 & 0 & 0\\
		1 & 0 & 0 & 0\\
		0 & 0 & 0 & 0\\ 
		0 & 0 & 0 & 0
	\end{pmatrix} \cdot
	\begin{pmatrix}
		1 & 0 & 0 & 0\\
		0 & 1 & 0 & 0\\
		0 & 0 & 0 & 0\\ 
		0 & 0 & 0 & 0
	\end{pmatrix} = \begin{pmatrix}
		0 & 1 & 0 & 0\\
		1 & 0 & 0 & 0\\
		0 & 0 & 0 & 0\\ 
		0 & 0 & 0 & 0
	\end{pmatrix} = K_1\\
	\mathbb Z_2 \text{ my beloved..}
\end{gather*}

\subsection{j)}

Taking a look at $(\Lambda_{xy})_{22},(\Lambda_{xy})_{23}, (\Lambda_{xy})_{32}, (\Lambda_{xy})_{33}$ separately:
\begin{gather}
	(\Lambda_{xy})_{22} = \left(\exp(-\omega K_3)\right)_{22} = 1 + 0 + \omega^2/2 + 0 + \omega^4/24 + 0 + O(\omega^6) = \cos(\omega)\\
(\Lambda_{xy})_{22} = (\Lambda_{xy})_{33}\\
	(\Lambda_{xy})_{23} = \left(\exp(-\omega K_3)\right)_{23} = 0 + \omega + 0 - \omega^3/6 + 0 +\omega^5/120 + O(\omega^6) = \sin(\omega)\\
	(\Lambda_{xy})_{23} = -(\Lambda_{xy})_{32}
\end{gather}
Because I was kind of confused myself for a second: the $1$ in the cosine terms comes from the identity matrix, which is why there is no one in the sine terms, and there are ones along the diagonal in the final matrix!

\subsection{k)}

\subsection{l)}

The matrices $K_i$ and the parameter $\chi$ are generators of the Lorentz boost, where $K_i$ decides the direction of the boost and where $\chi$ influences how strong this boost is. The matrices $S_i$ and the parameter $\omega$ are generators of the spacetime Rotation, where $S_i$ decides the axis of rotation and where $\omega$ influences how far this rotation goes.

\end{document}
