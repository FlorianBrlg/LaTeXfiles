\documentclass[]{scrartcl}

\usepackage{\string~"/LaTeX/StylePackage"}

\title{Exercise 2}
\author{Florian Bierlage}
\date{30.4.2025}


\begin{document}

\maketitle
\newpage
\tableofcontents
\newpage

\section{Exercise 1}
\subsection{a)}

The distance between two particles $d = \sqrt{V}$ where $V$ is the volume per particle.
$$
pV = N_A RT \rightarrow V, \;\;\; d= 3.4nm
$$
and for silicon
$$
\frac{\rho}{M_{Si}}=28u \Rightarrow V\;\;\; d = 0.27nm
$$
via the lattice constant $d = 0.5nm$.

\subsection{b)}
$$
V_{\text{Tube}} = \pi b^2 \cdot 1cm
$$
$$
\frac{V_\text{Tube}}{V} = 
\begin{cases}
	7.77\cdot 10^{23}b^2 m^{-2} & \text{ideal gas}
	1.6\cdot10^{27}b^2 m^{-2} & \text{Silicon}
\end{cases}
$$

\subsection{c)}

with $b=0.1nm$; $7700$ Air particles, and $1.6\cdot^{7}$ Silicon particles.

\subsection{d)}

$$
\sigma = \pi b^2 P
$$

\section{Exercise 2}

Reading from the figure, we get
$$
\diff{E}{x}=
\begin{cases}
	2.1 & \mu 	\\
	2.0 & \pi 	\\
	2.8 & p		\\
\end{cases}
$$
now to get $2\sigma$ we multiply all values by $2\cdot 5\%$.\\
the Proton is separated from the muon and the pion, but the pion and muon are very close to each other, so they are indistinguishable.

\section{Exercise 3}

\subsection{a)}

\begin{itemize}
	\item in the Order of $10MeV$ 
	\item Bremsstrahlung in the Order of 
	\item 
\end{itemize}

\subsection{b)}

$$
\left(\diff{E}{x}\right)_\text{rad} = - \frac{E}{X_0}
$$

$$
\frac{E}{X_0} = \sum_i \omega_i \frac{E}{X_{0i}}
$$
$$
\frac{1}{X_0} = \sum_i \omega_i \frac{1}{X_{0i}}
$$
where $X_0 = 43.2 \frac{g}{cm^2}$, or $X_0 = 32.5 \frac{1}{cm}$. 

\subsection{c)}

Al: 3.37\%, 8.89cm\\
Si: 0.21\%, 9.36cm


\end{document}

