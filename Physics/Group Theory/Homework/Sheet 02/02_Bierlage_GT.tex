!%TeX program = lualatex
\documentclass[]{scrartcl}

\usepackage{\string~"/LaTeX/StylePackage"}

\title{Group Theory - Sheet 2}
\author{}
\date{\today}


\begin{document}

\maketitle
\newpage
\tableofcontents
\newpage

\section{Groups of Order 6}

\subsection{a)}
Show that ever group of even order has an element of order 2

\textit{Proof: } Proof by construction. Let $G$ be a group of order $n$ such that there is a natural number $k$ which fulfils $2k = n$. Then there exists an element $g\in G$ such that $e = g^n = g^{2k}$. That means, that $e = g^{2k} = g^k \cdot g^k$, meaning that $g^k$ is an element of Order 2.

\subsection{b)}
Show that ever group of order 6 has an element of order 2

\textit{Proof: } Proof by construction. Let $G$ be a group of order 6. Let $g\in G$ such that $g^6 = e$. We know that $6 = 2\cdot2\cdot2$, hence $g^6 = g^2\cdot g^2\cdot g^2 = e$. Meaning that $g^2$ is an element of Order 3.

\subsection{c)}
let $a$ denote the order 3 element, let $b$ denote the order 2 element. Denote the identity element as
\begin{gather*}
	Id = a^3 = b^2
\end{gather*}
In this construction we notice another element of the Group:
\begin{gather*}
	d = a^2,\;\;\; Id = d^3
\end{gather*}
which is another order 3 element. We also see that
\begin{equation*}
	ad = da = Id,\;\;\; d^2 = a^4 = a^3 \cdot a = a
\end{equation*}
we can create the last two elements with $ab$ and $db$.
\begin{gather*}
	ab = c,\;\;\; db = f
\end{gather*}
Therefore we have the elements
\begin{gather*}
	Id = a^3 = b^2,\;\; d = a^2,\;\; c = ab,\;\; f = db = aab,\\
	G = \{Id, a, aa, b , ab ,aab\}
\end{gather*}


\subsection{d)}




\subsection{e)}
For $Z_6$ to be isomorphic to $Z_2\times Z_3$ there needs to be a mapping $f:Z_6 \rightarrow Z_2\times Z_3$ which preserves the Group structure. This can be shown from their cayley tables.

We can find this mapping from thinking about the properties these elements should have, that is, $(a,b)$ is isomorphic to $n$ if it has the same order. We deduce
\begin{gather*}
	f(0) = (0,0),\;\; f(1) = (1,1), \;\; f(2) = (0,2)\\
	f(3) = (1,0),\;\; f(4) = (0,1), \;\; f(5) = (1,2)
\end{gather*}

\section{Permutations}

$$
\sigma = 
\begin{pmatrix}
	1 & 2 & 3 & 4 & 5 & 6\\
	3 & 5 & 2 & 6 & 1 & 4
\end{pmatrix}
$$
\subsection{a)}

\begin{gather*}
	\sigma^2 = 
	\begin{pmatrix}
		1 & 2 & 3 & 4 & 5 & 5\\
		3 & 5 & 2 & 6 & 1 & 4
	\end{pmatrix} \cdot
	\begin{pmatrix}
		1 & 2 & 3 & 4 & 5 & 5\\
		3 & 5 & 2 & 6 & 1 & 4
	\end{pmatrix}
	=
	\begin{pmatrix}
		1 & 2 & 3 & 4 & 5 & 5\\
		3 & 5 & 2 & 6 & 1 & 4\\
		2 & 1 & 5 & 4 & 3 & 2
	\end{pmatrix}\\
	=
	\begin{pmatrix}
		1 & 2 & 3 & 4 & 5 & 6\\
		2 & 1 & 5 & 4 & 3 & 2
	\end{pmatrix}
\end{gather*}
to find the inverse we need to find a permutation such that $\sigma\cdot\sigma^{-1} = e$.
\begin{gather*}
	\text{let } \sigma^{-1} = 
	\begin{pmatrix}
		1 & 2 & 3 & 4 & 5 & 6\\
		5 & 3 & 1 & 6 & 2 & 4
	\end{pmatrix}
	\text{ then}\\
	\begin{pmatrix}
		1 & 2 & 3 & 4 & 5 & 6\\
		5 & 3 & 1 & 6 & 2 & 4
	\end{pmatrix} \cdot
	\begin{pmatrix}
		1 & 2 & 3 & 4 & 5 & 5\\
		3 & 5 & 2 & 6 & 1 & 4
	\end{pmatrix}
	=
	\begin{pmatrix}
		1 & 2 & 3 & 4 & 5 & 5\\
		3 & 5 & 2 & 6 & 1 & 4\\
		1 & 2 & 3 & 4 & 5 & 6
	\end{pmatrix}\\
	=
	\begin{pmatrix}
		1 & 2 & 3 & 4 & 5 & 6\\
		1 & 2 & 3 & 4 & 5 & 6
	\end{pmatrix} = e
\end{gather*}
Therefore, this is the inverse of $\sigma$.

\subsection{b)}
$$
\sigma = (1325)(64)
$$
\subsection{c)}

$$
\sigma = (1325)(64) = (13) (32) (25) (64)
$$

the permutation is even.

\section{Permutations and Feynmann Graphs}

I don't really know what this means right now, if I bother to figure it out before the deadline you won't be reading this

\section{Quaternions and Cayley's Theorem}

We will assign a number to every Quaternion Element:
\begin{gather*}
	1 = 1,\;\; -1 = 2,\;\; 3 = i,\;\; 4 = -i\\
	5 = j,\;\; 6 = -j,\;\; 7 = k,\;\; 8 = -k
\end{gather*}
And we will multiply the ordered set $(1,2,3,4,5,6,7,8)$ by each Quaternion to see its permutation. There will be 8 permutations in total.
\begin{gather*}
	1(1,2,3,4,5,6,7,8) = (1,2,3,4,5,6,7,8) \Rightarrow 1 =
	\begin{pmatrix}
		1 & 2 & 3 & 4 & 5 & 6 & 7 & 8\\
		1 & 2 & 3 & 4 & 5 & 6 & 7 & 8\\
	\end{pmatrix}\\
	-1(1,2,3,4,5,6,7,8) = (2,1,4,3,6,5,8,7) \Rightarrow -1 =
	\begin{pmatrix}
		1 & 2 & 3 & 4 & 5 & 6 & 7 & 8\\
		2 & 1 & 4 & 3 & 6 & 5 & 8 & 7\\
	\end{pmatrix}\\
	i(1,2,3,4,5,6,7,8) = (3,4,2,1,7,8,6,5) \Rightarrow i = 
	\begin{pmatrix}
		1 & 2 & 3 & 4 & 5 & 6 & 7 & 8\\
		3 & 4 & 2 & 1 & 7 & 8 & 6 & 5\\
	\end{pmatrix}\\
	-i(1,2,3,4,5,6,7,8) = (4,3,1,2,8,7,5,6) \Rightarrow -i =
	\begin{pmatrix}
		1 & 2 & 3 & 4 & 5 & 6 & 7 & 8\\
		4 & 3 & 1 & 2 & 8 & 7 & 5 & 6\\
	\end{pmatrix}\\
	j(1,2,3,4,5,6,7,8) = (5,6,8,7,2,1,3,4) \Rightarrow j =
	\begin{pmatrix}
		1 & 2 & 3 & 4 & 5 & 6 & 7 & 8\\
		5 & 6 & 8 & 7 & 2 & 1 & 3 & 4\\
	\end{pmatrix}\\
	-j(1,2,3,4,5,6,7,8) = (6,5,7,8,1,2,4,3) \Rightarrow -j =
	\begin{pmatrix}
		1 & 2 & 3 & 4 & 5 & 6 & 7 & 8\\
		6 & 5 & 7 & 8 & 1 & 2 & 4 & 3\\
	\end{pmatrix}\\
	k(1,2,3,4,5,6,7,8) = (7,8,5,6,4,3,2,1) \Rightarrow k =
	\begin{pmatrix}
		1 & 2 & 3 & 4 & 5 & 6 & 7 & 8\\
		7 & 8 & 5 & 6 & 4 & 3 & 2 & 1\\
	\end{pmatrix}\\
	-k(1,2,3,4,5,6,7,8) = (8,7,6,5,3,4,1,2) \Rightarrow -k=
	\begin{pmatrix}
		1 & 2 & 3 & 4 & 5 & 6 & 7 & 8\\
		8 & 7 & 6 & 5 & 3 & 4 & 1 & 2\\
	\end{pmatrix}		
\end{gather*}
So, the subgroup that $Q_8$ represents is (in cycle notation):
$$
\left\{(), (12)(34)(56)(78), (1324)(5768), (1423)(5867), (1526)(3847), (1625)(3748), (1728)(3546), (1827)(3645)\right\}
$$

\section{The Dihedral Group $D_4$}

\subsection{a)}
The Dihedral Group is the Group $D_4 = \{e,r,r^2,r^3,s,sr,sr^2,sr^3\}$ where $r$ signifies a $\pi/2$ rotation, and $s$ signifies a reflection. The multiplication table goes as follows
$$
\begin{array}{|c||c|c|c|c|c|c|c|c|}\hline
    \cdot & e & r & r^2 & r^3 & s & sr & sr^2 & sr^3 \\
    \hline\hline
    e & e & r & r^2 & r^3 & s & sr & sr^2 & sr^3 \\\hline
    r & r & r^2 & r^3 & e & sr & sr^2 & sr^3 & s \\\hline
    r^2 & r^2 & r^3 & e & r & sr^2 & sr^3 & s & sr \\\hline
    r^3 & r^3 & e & r & r^2 & sr^3 & s & sr & sr^2 \\\hline
    s & s & sr^3 & sr^2 & sr & e & r^3 & r^2 & r \\\hline
    sr & sr & s & sr^3 & sr^2 & r & e & r^3 & r^2 \\\hline
    sr^2 & sr^2 & sr & s & sr^3 & r^2 & r & e & r^3 \\\hline
    sr^3 & sr^3 & sr^2 & sr & s & r^3 & r^2 & r & e \\\hline
\end{array}
$$

\subsection{b)}

Two elements $a,b$ are conjugate to each other if there is an element $x$ such that
$$
b = xax^{-1}
$$
A conjugacy class is the Group of all elements conjugate to each other.
\begin{gather}
	\{e\},\;\;\text{The identity is only conjugate to itself}\nonumber\\
	\{r,r^3\},\;\; s r s^{-1} = s s r^3 = r^3\nonumber\\
	\{r^2\},\;\; \text{The $\pi$ rotation is only conjugate to itself}\nonumber\\
	\{s, sr^2\},\;\; r s r^{-1} = r (sr^3) = sr^2\nonumber\\
	\{sr, sr^3\},\;\; r sr r^{-1} = r s = sr^3\nonumber
\end{gather}
One way to view a conjugacy class is that if two symmetries are equivalent if you change your perspective (how you look at the object) then they are in one conjugacy class. This for example gains physical interpretations with Lorentz Transformations like $U \phi(x) U^\dagger = \phi(\Lambda^{-1}x)$, or other similar Transformations.

\subsection{c)}
From Lagrange's Theorem we know that the order of the subgroup of a group must be a divisor of the group's order. We also know that a Group must contain the identity element $e$. From this we can gather that we have the subgroups
\begin{gather*}
	\{e,r^2\},\;\; \{e,r,r^2,r^3\},\;\;\{e,r^2,s,sr^2\}\\
	\{e,r^2,sr,sr^3\},\;\;\{e,r,r^2,r^3,sr,sr^2,sr^3\}
\end{gather*}
As these Groups were formed by individual conjugate classes, they are all normal. There are also Subgroups with the reflections as generators
\begin{gather*}
	\{e,s\},\;\;\{e,sr\},\;\;\{e,sr^2\},\;\;\{e,sr^3\}
\end{gather*}
These are not normal.

\subsection{d)}
A Group $G$ can be written as a direct product of two of its subgroups $H, P$ if
\begin{itemize}
	\item The intersection of $H$ and $P$ is trivial\\
	\item Every Element of $G$ can be uniquely determined from elements in $H$ and $P$
	\item Both $H$ and $P$ are normal in $G$
\end{itemize}
As all our normal subgroups share the element $r^2$, and thus its intersections are not trivial, we cannot construct $D_4$ from a direct product of its subgroups.



\end{document}

