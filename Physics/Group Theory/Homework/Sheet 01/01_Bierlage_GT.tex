\documentclass[]{scrartcl}

\usepackage{\string~"/LaTeX/StylePackage"}

\title{Group Theory - Sheet 1}
\author{Florian}
\date{10.10.2024}


\begin{document}

\maketitle
\newpage
\tableofcontents
\newpage

\section{Rotation Symmetries of a cube}

\subsection{a)}

The identity is the 1st invariance.\\
Then there is an Invariance for every $\pi/2$ turn around any axis. That is, taking the example of the $z$ axis, we can turn the cube $\pi/2$, $\pi$, $3\pi/2$ and get an invariance. $2\pi$ is not counted as it is equivalent to the Identity.\\
These are 3 Invariances for 3 unique axes, meaning that there are 9 Invariances.\\
There is an Invariance for every $2\pi/3$ turn around any vertex. That is for an axis of the form $(b_x, b_y, b_z)$ where $b_i$ is a basis vector. We can turn the cube $2\pi/3$ and $4\pi/3$ and get an Invariance.\\
These are 2 Invariances for 4 unique axes, meaning there are 8 Invariances.\\
There is an invariance for every $\pi$ turn around any edge. That is for an axis of the form $(b_x, b_y, 0)$.\\
There is 1 Invariance for 6 unique axes, meaning there are 6 Invariances.

In total we have $6+8+9+1 = 24$ Rotational invariances.


\section{Chiral Group Axioms}

\subsection{a)}
In the lecture:
\begin{itemize}
	\item Associativity: $(a_i * a_j) * a_k = a_i * (a_j * a_k)$
	\item Neutral element: $\exists e\in G\forall a\in G(e*a = a*e = a)$
	\item Inverse element: $\forall a \exists a^{-1}(a*a^{-1} = a^{-1}*a=e)$
\end{itemize}
We want to prove that these are equivalent to the Definitions on the Homework sheet. Associativity is trivially equivalent.
\subsubsection{Neutral Element}
$$
\exists e\forall a\in G (e*a = a*e = a) \Leftrightarrow \exists e\in G \forall a\in G(a * e = a)
$$
So, we want to see that $a*e = a$ somehow implies $e*a = a$ for this to be equivalent.

\textit{Proof: } Assume that $\exists e\in G \forall a\in G(a*e = a)$. Assume that $\forall a\in G\exists a^{-1}\in G(a*a^{-1} =e)$. We now want to see that $a*e = e*a$. We know that two elements are the same if they are equal if connected with another element $a*e*d = e*a*d$. We will show this by picking $d = a^{-1}$.
\begin{gather}
	a*e*a^{-1} = e*a*a^{-1}\\
	a*a^{-1} = e*e\\
	e = e
\end{gather}
So, $a*e = e$ and $a*a^{-1}=e$ implies $e*a = a*e$. And as equality is transitive, $e = a * e = e* a$. Meaning that the axioms are equivalent, if swapped.

\subsubsection{Inverse Element}
$$
\forall a\exists a^{-1}(a*a^{-1} = a^{-1}*a = e) \Leftrightarrow \forall a\exists a^{-1}(a*a^{-1}=e)
$$
We then want to see if $a*a^{-1}=e$ implies $a^{-1}*a=e$. We do the same as we did previously, showing that $a * a^{-1} = a^{-1} * a$

\textit{Proof:} Assume that $\forall a\exists a^{-1}(a*a^{-1}=e)$ and $\exists e\forall a(a*e = a)$. Through the previously proven statement we can also Assume that $\exists e \forall a(e*a = a)$. 
\begin{gather}
	a * a^{-1} = a^{-1} * a\\
	a * a * a^{-1} = a * a^{-1} * a\\
	a * e = e * a\\
	a = a
\end{gather}
So $a*a^{-1} = e$ and $e*a = a*e = a$ imply that $a*a^{-1} = a^{-1}*a$, and through the transitive property of equality we have $e = a*a^{-1} = a^{-1}*a$.
\subsubsection{Why only one direction}
I only proved the "$\Leftarrow$" direction in both cases, that is because the "$\Rightarrow$" direction is trivial, as the right hand side is included in the left hand side.
\subsubsection{Alternative direction}
These axioms would still be equivalent with $e*a=a$ or $a^{-1}*a=e$, however the proofs would have to be adjusted slightly.

\subsection{b)}
We want to show that $e^{-1} = e$, so that $e^{-1}$ acts on elements as $e$ would.

\textit{Proof: } Assume $\exists e\forall a (e*a=a)$. As this holds for all $a$, it also holds for $e$. So we know that $e*e = e$. Then, from the definition of inverses we know that $a*a^{-1} = e$, particularly we know that $e*e^{-1}=e$, setting them equal we get $e * e = e* e^{-1}$, from this we find that $e = e^{-1}$.

\section{Groups, Yes or No?}
\subsection{a)}
It does not obey associativity:
\begin{gather}
	\begin{pmatrix}
		a_1 \\ a_2 \\ a_3
	\end{pmatrix}\times \left(
		\begin{pmatrix}
			b_1 \\ b_2 \\ b_3
		\end{pmatrix} \times
		\begin{pmatrix}
			c_1 \\ c_2 \\ c_3
		\end{pmatrix}
	\right)
	=
	\left(
\begin{pmatrix}
		a_1 \\ a_2 \\ a_3
	\end{pmatrix}\times
		\begin{pmatrix}
			b_1 \\ b_2 \\ b_3
		\end{pmatrix}
		\right)\times
		\begin{pmatrix}
			c_1 \\ c_2 \\ c_3
		\end{pmatrix}
	\\
	\begin{pmatrix}
		a_1\\a_2\\a_3
	\end{pmatrix}\times
	\begin{pmatrix}
		b_2c_3 - b_3c_2 \\ b_3c_1 - b_1c_3\\ b_1c_2 - b_2c_1
	\end{pmatrix}
	=
	\begin{pmatrix}
		a_2b_3 - a_3b_2 \\ a_3b_1 - a_1b_3\\ a_1b_2 - a_2b_1
	\end{pmatrix}\times
	\begin{pmatrix}
		c_1 \\ c_2 \\ c_3
	\end{pmatrix}
	\\
	\text{With the first entry on both sides being:}\nonumber\\
	(a_3b_1 - a_1b_3)c_3 - (a_1b_2 - a_2b_1)c_2 = a_2(b_1c_2 - b_2c_1) - a_3(b_3c_1 - b_1c_3)
\end{gather}
Which is not equal as the left side contains $a_1$ while the right side does not.
\subsection{b)}
The set lacks an inverse for $3$. (The neutral element is $1$)
\begin{gather}
	3\cdot 1(\text{mod}3) = 3 (\text{mod}3)\\
	3\cdot 2(\text{mod}3) = 6(\text{mod}3) = 3(\text{mod}3)\\
	3\cdot3(\text{mod}3) = 9(\text{mod}3) = 3(\text{mod}3)
\end{gather}

\subsection{c)}

\subsubsection{Neutral Element}
The neutral element is $f_1$.
\begin{gather}
	f_1(f_1(x)) = x,\;\;\;\; f_1(f_2(x)) = -x\\
	f_1(f_3(x)) = \frac{1}{x}, \;\;\;\; f_1(f_4(x)) = \frac{-1}{x}
\end{gather}

\subsubsection{Inverse Element}
The inverse element of every element is itself.
\begin{gather}
	f_1(f_1(x)) = x\\
	f_2(f_2(x)) = -(-x) = x\\
	f_3(f_3(x)) = \frac{1}{1/x} = x\\
	f_4(f_4(x)) = \frac{-1}{-1/x} = x
\end{gather}

\subsubsection{Associativity}
I will prove associativity of composition which is more general than the composition of these 4 elements.

\textit{Proof: } let $f:X\rightarrow X$, $g:X\rightarrow X$, $h:X\rightarrow X$ be functions. Then
\begin{gather}
	f\cdot(g\cdot h) = (f\cdot g)\cdot h\\
	f\cdot(g(h(x))) = (f(g(x))\cdot h\\
	f(g(h(x))) = f(g(h(x)))
\end{gather}
Both sides are equivalent, proving associativity.\\\\
Hence, The functions form a Group. More generally, bijective functions form a Group.
\subsection{Simple Statements}
\subsubsection{a)}
$$
\forall a\in G(aG = G),\;\; aG := \{ab|b\in G\}
$$
\textit{Proof: }
As a Group underlies the closure property, $\forall a,b \in G(ab\in G)$. So, all elements of $aG$ are elements of $G$, meaning $aG\subset G$. Every element $b$ can be represented as $b = aa^{-1}b$. As $a^{-1}b\in G$, $aa^{-1}b = b \in aG$. So, $aG \subset G$.

As $G \subset aG$ and $aG \subset G$, $G = aG$.
\subsubsection{b)}
$$
\forall a\in G(a^2 = e) \Rightarrow \forall a,b \in G(ab = ba)
$$
\textit{Proof: }Assume $\forall x\in G(x^2 = e)$. Let $a,b \in G$.
\begin{gather}
	ab = ba\\
	\text{Multiplying $ab$ from the left}\nonumber\\
	abab = abba\\
	(ab)^2 = aea\\
	\text{using $x^2 = e$}\nonumber\\
	e = aa\\
	e = e
\end{gather}
Which proves the statement.

\subsection{c)}
$Z_2\times Z_4 = \{a_1b_1, a_1b_2, a_1b_3, a_1b_4, a_2b_1, a_2b_2, a_2b_3, a_2b_4\}$\\
$Z_8 = \{c_1, c_2, c_3, c_4, c_5, c_6, c_7, c_8\}$\\\\
\begin{tabular}{|c||c|c|c|c|c|c|c|c|}
	\hline
	& $a_1b_1$ & $a_1b_2$ & $a_1b_3$ & $a_1b_4$ & $a_2b_1$ & $a_2b_2$ & $a_2b_3$ & $a_2b_4$\\\hline\hline
	$a_1b_1$ & $a_1b_1$ & $a_1b_2$ & $a_1b_3$ & $a_1b_4$ & $a_2b_1$ & $a_2b_2$ & $a_2b_3$ & $a_2b_4$ \\\hline
	$a_1b_2$ & $a_1b_2$ & $a_1b_3$ & $a_1b_4$ & $a_1b_1$ & $a_2b_2$ & $a_2b_3$ & $a_2b_4$ & $a_2b_1$ \\\hline
	$a_1b_3$ & $a_1b_3$ & $a_1b_4$ & $a_1b_1$ & $a_1b_2$ & $a_2b_3$ & $a_2b_4$ & $a_2b_1$ & $a_2b_2$\\\hline
	$a_1b_4$ & $a_1b_4$ & $a_1b_1$ & $a_1b_2$ & $a_1b_3$ & $a_2b_4$ & $a_2b_1$ & $a_2b_2$ & $a_2b_3$ \\\hline
	$a_2b_1$ & $a_2b_1$ & $a_2b_2$ & $a_2b_3$ & $a_2b_4$ & $a_1b_1$ & $a_1b_2$ & $a_1b_3$ & $a_1b_4$ \\\hline
	$a_2b_2$ & $a_2b_2$ & $a_2b_3$ & $a_2b_4$ & $a_2b_1$ & $a_1b_2$ & $a_1b_3$ & $a_1b_4$ & $a_1b_1$ \\\hline
	$a_2b_3$ & $a_2b_3$ & $a_2b_4$ & $a_2b_1$ & $a_2b_2$ & $a_1b_3$ & $a_1b_4$ & $a_1b_1$ & $a_1b_2$ \\\hline
	$a_2b_4$ & $a_2b_4$ & $a_2b_1$ & $a_2b_2$ & $a_2b_3$ & $a_1b_4$ & $a_1b_1$ & $a_1b_2$ & $a_1b_3$ \\\hline
\end{tabular}



\begin{tabular}{|c||c|c|c|c|c|c|c|c|}
	\hline
	& $c_1$ & $c_2$ & $c_3$ & $c_4$ & $c_5$ & $c_6$ & $c_7$ & $c_8$\\\hline\hline
	$c_1$ & $c_1$ & $c_2$ & $c_3$ & $c_4$ & $c_5$ & $c_6$ & $c_7$ & $c_8$ \\\hline
	$c_2$ & $c_2$ & $c_3$ & $c_4$ & $c_5$ & $c_6$ & $c_7$ & $c_8$ & $c_1$ \\\hline
	$c_3$ & $c_3$ & $c_4$ & $c_5$ & $c_6$ & $c_7$ & $c_8$ & $c_1$ & $c_2$ \\\hline
	$c_4$ & $c_4$ & $c_5$ & $c_6$ & $c_7$ & $c_8$ & $c_1$ & $c_2$ & $c_3$ \\\hline
	$c_5$ & $c_5$ & $c_6$ & $c_7$ & $c_8$ & $c_1$ & $c_2$ & $c_3$ & $c_4$ \\\hline
	$c_6$ & $c_6$ & $c_7$ & $c_8$ & $c_1$ & $c_2$ & $c_3$ & $c_4$ & $c_5$ \\\hline
	$c_7$ & $c_7$ & $c_8$ & $c_1$ & $c_2$ & $c_3$ & $c_4$ & $c_5$ & $c_6$ \\\hline
	$c_8$ & $c_8$ & $c_1$ & $c_2$ & $c_3$ & $c_4$ & $c_5$ & $c_6$ & $c_7$ \\\hline
\end{tabular}

As we can see, the neutral elements $a_1b_1$, $c_1$ are in completely different places. Meaning that if we tried to establish an isomorphic mapping, where for example $c_1 = a_1b_1, c_2 = a_1b_2, \cdots, c_8 = a_2b_2$ then we would have $c_1c_1 = e = a_1b_1 a_1b_1$, however we can see from the first table that $a_1b_2 a_1b_4 = e$, meaning that $c_2c_4 = e$, but as you can see from the second table this isn't true. So there is no isomorphic mapping that preserves the Group structure.\\\\
This could also be proven by comparing the Order of each element of each Group, which is the smallest $n$ such that $g^n = e$ for an element $g\in G$.
\end{document}

