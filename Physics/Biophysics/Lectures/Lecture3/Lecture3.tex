\documentclass[]{scrartcl}

\usepackage{\string~"/LaTeX/StylePackage"}

\title{Biophysics - Lecture 3}
\author{}
\date{3.9.2024}


\begin{document}

\maketitle
\newpage
\tableofcontents
\newpage

\section{The concept of biological macromolecules}

There are 4 main Groups
\begin{itemize}
	\item Proteins (Amino Acids)
	\item Nucleic Acids (Nucleotides)
	\item Polysaccarides (Carbohydrates)
	\item Lipids (Fatty Acids, Glycerol, etc)
\end{itemize}

Lipids consist of a hydrophobic tail and a hydrophillic head.

Biomolecules - some characteristics
\begin{itemize}
	\item Polymers
		\begin{itemize}
			\item Long chains of limited types of small molecules
			\item Water soluble
		\end{itemize}
	\item Lipids
		\begin{itemize}
			\item Insoluble in water, or amphiphatic
		\end{itemize}
\end{itemize}

Proteins are structured in 4 main parts
\begin{itemize}
	\item Primary Structure
	\item Secondary Structure
	\item Tertiary Structure
	\item Quaternary Structure
\end{itemize}
\subsection{Primary Structure}
The linear sequence of amino acids. Proteins are built from amino acids, forming polypeptide chains.

An Amino acid contains the Amino ($NH_2$) group, and the carboxyl ($COOH$) group.

\subsection{Secondary Structure}
The local spatial arrangement of the polypeptide chain

\subsection{Tertiary Structure}
The three-dimensional structure of the entire polypeptide chain

\subsection{Quaternary structure}

\subsubsection{Hemoglobin}
4 almost identical parts
\begin{itemize}
	\item Two $\alpha$ chains (each 141 amino acids)
	\item Two $\beta$ chains (146 amino acids)
	\item each containing an Fe-Atom.
\end{itemize}
$O_2$ or $CO_2$ bins to one Fe site. The molecular structure changes (Opens up). more $O_2$ or $CO_2$ may bind to other Fe sites in the hemoglobin. 


\section{Basics on different families of chemical bonding}



\end{document}

