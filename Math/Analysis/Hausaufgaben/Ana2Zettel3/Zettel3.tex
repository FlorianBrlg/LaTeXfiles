\documentclass[]{scrartcl}

\usepackage{\string~"/LaTeX/StylePackage"}

\title{Zettel 3}
\author{}
\date{xx.xx.2025}


\begin{document}

\maketitle
\newpage
\tableofcontents
\newpage

\section{}

\section{}

\section{}
\subsection{(i)}

Sei $K = \cup_i k_i \subset X$ mit $k_i$ kompakten Mengen. Da $k_i$ kompakt gibt es 
jeweils eine offene Überdeckung $U_i = \cup_j\left(\{U_j\}_i\right)$, s.d. $k_i\subset U_i$.
Wir finden jeweils eine Teilmenge $M_i$ sodass $M_i$ eine endliche Teilmenge von $U_i$
ist, welche $k_i$ bedeckt.\\
Jetzt ist $\cup_i U_i$ eine offene Überdeckung von $K$ und $\cup_i M_i$ eine endliche Teilmenge von $cup_i U_i$.\\
Da dies für beliebige $K=\cup_i k_i$ gilt ist die Aussage korrekt.

\subsection{(ii)}
Sei $K_n = [1/n, 1]$. Diese Mengen sind kompakt. Jetzt sei $K = \cup_n^\infty K_n$, diese Menge ist
$$
(0, 1]
$$
Diese Menge ist offen in $\mathbb R$, und somit nicht kompakt. Die Aussage ist somit falsch.

\subsection{(iii)}
$\mathbb R$ ist vollständig, allerdings ist $\mathbb R$ offen und somit nicht kompakt.

\subsection{(iv)}
Sei $X$ ein kompakter Raum. In kompakten Räumen hat jede Folge eine Konvergierende Teilfolge. 
Sei $x_n$ eine Cauchy Folge, dann hat $x_n$ eine konvergierende Teilfolge. 
In einem metrischen Raum konvergieren alle Cauchy Folgen die eine konvergierende Teilfolge haben, 
somit konvergieren alle Cauchy Folgen und $X$ ist Vollständig

\subsection{(v)}
Sei $a_n$ eine Folge in $\ell^2$. Sei $a_n = e_n$ die $n$-te Basis. $\ell^2$ ist unendlich 
dimensional, also hat $a_n$ keinen Grenzwert, und auch keine konvergierende Teilfolge. 
Somit ist $\ell^2$ nicht kompakt.

\section{}

\subsection{(i)}
Sei $A\subset X$ abgeschlossen. Sei $U_i$ eine offene Überdeckung von $A$. Da $A$ in $X$ abgeschlossen ist, ist $X\backslash A$ offen. Dann ist $U_i \cap X\backslash A$ eine offene Überdeckung von $X$. Da $X$ Kompakt ist, existiert eine endliche Teilüberdeckung $\{U_1,\cdots, U_n\}\cap X\backslash A$.
Dann ist $\{U_1,\cdots, U_n\}$ eine offene Teilüberdeckung von $A$. Somit ist $A$ kompakt.

\subsection{(ii)}

Sei $U$ eine offene Überdeckung von $f(K)$. Weil $f$ stetig ist, ist $f^{-1}(u)$ offen für alle $u\in U$.
Die Menge $\{f^{-1}(u)| u\in U\}$ ist eine offene Überdeckung von $K$, 
weil für alle $x\in K$ $f(x)$ in einem $u\in U$ sein muss.\\
Weil $K$ kompakt ist, hat es eine endliche Überdeckung $\{f^{-1}(u_1), \cdots, f^{-1}(u_n)\}$.
Daraus folgt dass $\{u_1, \cdots, u_n\}$ eine endliche Überdeckung von $f(K)$ ist.

\end{document}

