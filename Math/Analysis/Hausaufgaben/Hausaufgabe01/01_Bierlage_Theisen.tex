\documentclass[]{scrartcl}

\usepackage{\string~"/LaTeX/StylePackage"}

\title{Hausaufgabe 1}
\author{Florian, Jan}
\date{\today}


\begin{document}

\maketitle
\newpage
\tableofcontents
\newpage

\section{Summenformeln}

\subsection{a)}

\textit{Beweis:} Beweis durch Induktion. Wir beweisen den Basisfall $n=0$.
\begin{gather}
	\sum_{k=1}^0 k^2 = \frac{0(0+1)(2\cdot0+1)}{6}\\
	0 = 0
\end{gather}
Wir nehmen jetzt an, dass die Aussage für $n$ stimmt, und beweisen die Aussage für $n \mapsto n+1$.
\begin{gather}
	\sum_{k=1}^{n+1} k^2 = \sum_{k=1}^n k^2 + (n+1)^2\\
	= \frac{n(n+1)(2n+1)}{6} + \frac{6n^2 + 12n + 6}{6}\\
	= \frac{(n^2+n)(2n+1)}{6} + \frac{6n^2 + 12n + 6}{6}\\
	= \frac{2n^3 + n^2 + 2n^2 + n + 6n^2 + 12n + 6}{6} = \frac{2n^3 + 9n^2 + 13n + 6}{6}\\
	\text{Was wir bekommen wollen ist: } \frac{(n+1)(n+2)(2n+3)}{6} \text{ also berechnen:}\nonumber\\
	\frac{(n+1)(n+2)(2n+3)}{6} = \frac{(n^2 + 3n + 2)(2n+3)}{6}\\
	= \frac{2n^3 + 3n^2 + 6n^2 + 9n + 4n + 6}{6} = \frac{2n^3 + 9n^2 +13n + 6}{6}
\end{gather}
Das ist genau das was wir haben wollten. Da Die Aussage für $n = 0$ stimmt, und für $n\mapsto n+1$ stimmt, falls die Aussage für $n$ stimmt, stimmt die Aussage für ganz $\mathbb{N}$.

\subsection{b)}
\textit{Beweis: } Beweis durch Induktion. Wir beweisen den Basisfall $n = 0$.
\begin{gather}
	\sum_{k=1}^0 \frac{1}{k(k+1)} = \frac{0}{0 + 1}\\
	0 = 0
\end{gather}
Wir nehmen jetzt an, dass die Aussage für $n$ stimmt, und beweisen die Aussage für $n\mapsto n+1$.
\begin{gather}
	\sum_{k=1}^{n+1} \frac{1}{k(k+1)} = \sum_{k=1}^n \frac{1}{k(k+1)} + \frac{1}{(n+1)(n+2)}\\
	= \frac{n}{n+1} + \frac{1}{(n+1)(n+2)} = \frac{n^2 + 2n + 1}{(n+1)(n+2)}\\
	= \frac{(n+1)^2}{(n+1)(n+2)} = \frac{n+1}{n+2}
\end{gather}
Das ist genau das, was wir haben wollten. Da die Aussage für $n=0$ stimmt, und für $n\mapsto n+1$ stimmt, falls die Aussage für $n$ stimmt, stimmt die Aussage für ganz $\mathbb{N}$

\section{Fibonacci Zahlen}
\subsection{a)}
\textit{Beweis: } Beweis durch Induktion, Beweise $n = 0$ und $n=1$.
\begin{gather}
	F_1 = \sum_{i=0}^ 0
	\begin{pmatrix}
		0 - i \\ i
	\end{pmatrix} = 1
\end{gather}
\begin{gather}
	F_2 = \sum_{i=0}^1 {1-i\choose i} = {1\choose 0} + {0 \choose 1} = 1 + 0 = 1
\end{gather}
Wir nehmen jetzt an, dass die Aussage für $n$ und $n-1$ stimmt, und beweisen die Aussage für $n\mapsto n+1$.
\begin{gather}
	F_{n+2}\sum_{i=0}^{n+1} {n+1-i\choose i} = \sum_{i = 0}^n {n+1-i \choose i} + \underbrace{\frac{0}{n+1}}_{=0}\\
	= \sum_{i=0}^n {n-i\choose i}+{n-i \choose i-1} = \sum_{i=0}^n {n-i \choose i} + \sum_{i=1}^{n} {n-i \choose i-1}\\
	= \sum_{i=0}^n {n-i \choose i} + \sum_{i=0}^{n-1} {n-1-i\choose i}\\
	= F_{n+1} + F_n
\end{gather}
Da die Aussage für $n+1$ gilt, falls sie für $n$ und $n-1$ gilt, und die Aussage für $n=0$ und $n=1$ gilt, gilt die Aussage für ganz $\mathbb{N}$

\subsection{b)}
\textit{Beweis: } Beweis durch Induktion Beweise $n=0$ und $n=1$.
\begin{gather}
	F_0 = \frac{1}{\sqrt 5}\left(\left(\frac{1}{2}(1+\sqrt 5)\right)^0 - \left(\frac{1}{2}(1-\sqrt5)\right)^0\right) = \frac{1}{\sqrt5}(1-1) = 0\\
	F_1 = \frac{1}{\sqrt5}\left(\left(\frac{1}{2}(1+\sqrt5)\right)^1 - \left(\frac{1}{2}(1-\sqrt5)\right)\right)\\
	= \frac{1}{\sqrt5}\left(\frac{1}{2}2\sqrt5\right) = 1
\end{gather}
Nehme an dass die Aussage für $n$ und $n-1$ stimmt, und beweise die Aussage für $n+1$.
\begin{gather}
	F_{n+1} = \frac{1}{\sqrt5}\left(a^n - b^n + a^{n-1} - b^{n-1}\right)
\end{gather}
Es gilt außerdem, dass $a^2 - a - 1 = b^2 - b - 1 = 0$ ist. Wir entnehmen also folgendes für $a,b$:
\begin{gather}
	a^2 = a + 1\\
	\Rightarrow a^3 = a^2 + 1\\
	\Rightarrow a^3 = (a + 1) + a = 2a + 1\\
	\Rightarrow a^4 = 2a^2 + a = 2a + 2 + a = 3a + 2
\end{gather}
\subsubsection{Lemma:}
Es folgt das Lemma:\\
Seien $a,b\in\mathbb{R}$ Lösungen für $x^2 = x +1$ und $n\geq2$, dann
\begin{gather}
	a^n = F_n a + F_{n-1}\\
	b^n = F_n b + F_{n-1}
\end{gather}
\textit{Beweis: } Sei $x$ eine Lösung für $x^2 = x +1$. Beweis per Induktion. Für $n=2$ bis $n=4$ vorher bewiesen. Nehme an dass die Aussage für $n$ gilt und beweise für $n+1$.
\begin{gather}
	x^n = F_n x + F_{n-1}\\
	\Rightarrow x^{n+1} = F_n x^2 + F_{n-1}x = F_n (x+1) + F_{n-1}x\\
	\Rightarrow x^{n+1} = (F_n + F_{n-1})x + F_n = F_{n+1}x + F_n
\end{gather}
Da die Aussage für $n=2$ gilt, und für $n+1$ gilt, falls sie für $n$ gilt, gilt die Aussage für alle $n\geq2$.\\\\
Für $F_{n+1}$ heißt dies folgendes:
\begin{gather}
	F_{n+1} = \frac{1}{\sqrt5}\left((F_na + F_{n-1} + F_{n-1}a + F_{n-2}) - (F_nb + F_{n-1} + F_{n-1}b + F_{n-2})\right)\\
	= \frac{1}{\sqrt5}\left((F_{n+1}a + F_n) - (F_{n+1}b + F_n)\right)\\
	= \frac{1}{\sqrt5}(a^{n+1}-b^{n+1})
\end{gather}
Da die Aussage für $n = 0$ und $n = 1$ gilt, und für $n+1$ gilt falls sie für $n$ und $n-1$ gilt, gilt sie für ganz $\mathbb N$.

\section{Geometrische Reihe}

\subsection{a)}
\textit{Beweis: }Beweis durch Induktion. Beweise den Basisfall $n=0$.
\begin{gather}
	(a-b)\sum_{i=0}^0 a^ib^{n-i}\\
	= (a-b)\cdot 1 = a-b = a^1 - b^1
\end{gather}
Wir nehmen jetzt an, dass die Aussage für $n$ stimmt, und beweisen die Aussage für $n\mapsto n+1$.
\begin{gather}
	(a-b)\sum_{i=0}^{n+1} a^ib^{n+1-i} = (a-b)\left(\sum_{i=0}^{n} a^ib^{n+1-i} + a^{n+1}b^0\right)\\
	= (a-b)\left(\sum_{i = 0}^{n}a^ib^{n-i}\cdot b + a^{n+1}\right)\\
	= b(a^{n+1} - b^{n+1}) + (a-b)a^{n+1}\\
	= ba^{n+1} - b^{n+2} + a^{n+2} - ba^{n+1}\\
	= a^{n+2} - b^{n+2}
\end{gather}
Das ist genau das, was wir haben wollten. Da die aussage für $n=0$ stimmt, und für $n\mapsto n+1$ stimmt, falls sie für $n$ stimmt, stimmt die Aussage für ganz $\mathbb{N}$.

\subsection{b)}

\textit{Beweis: }Sei $n\geq 1$ und $x\neq 1$
\begin{gather}
	(1-x)\sum_{i=0}^n x^i = \sum_{i=0}^n x^i - \sum_{i=0}^n x^{i+1}\\
	= \sum_{i=0}^n x^i - \sum_{i=1}^{n+1} x^i = 1 + x^{n+1}\\
	\text{hier ist es wichtig, dass $x-1 \neq 0$.} \nonumber\\
	\Rightarrow \sum_{i=0}^n x^i = \frac{1+x^{n+1}}{1-x}
\end{gather}

\section{Ein Operator auf $\mathbb{R}$}

Sei die operation $*$ definiert als
$$
x*y = x + y + \frac{xy}{\lambda},\;\;\; \lambda\in\mathbb R \backslash \{0\}
$$
\subsection{Kommutativ:}
\begin{gather}
	x*y = x + y + \frac{xy}{\lambda}\\
	y*x = y + x + \frac{yx}{\lambda} = x + y \frac{xy}{\lambda}
\end{gather}
Kommutativität kommt durch die kommutativität der benutzten operationen.
\subsection{assoziativität}
\begin{gather}
	x*(y*z) = (x*y)*z\\
	\Rightarrow x*(y + z + \frac{yz}{\lambda}) = (x+y+\frac{xy}{\lambda}) * z\\
	\Rightarrow x + y + z + \frac{yz}{\lambda} + \frac{xy + xz + x\frac{yz}{\lambda}}{\lambda}
	= x + y + \frac{xy}{\lambda} + z + \frac{xz + yz + \frac{xy}{\lambda}z}{\lambda}\\
	\Rightarrow x+y+z+ \frac{yz + xy + xz}{\lambda}+\frac{xyz}{\lambda^2} = x+y+z+ \frac{xy + xz + yz}{\lambda} + \frac{xyz}{\lambda^2}
\end{gather}
Somit sind beide Seiten gleich und die Operation ist Assoziativ.
\subsection{Eins}
\begin{gather}
	x + y + \frac{xy}{\lambda} = x\\
	y + \frac{xy}{\lambda} = 0\\
	y(1 + \frac{x}{\lambda}) = 0\\
	y = 0
\end{gather}
Die Eins ist $0$, weil $x*0 = x$.

\subsection{Inverse}
\begin{gather}
	x * y = x + y + \frac{xy}{\lambda} = 0\\
	x(y + \frac{y}{\lambda}) = 0
\end{gather}
Dies hat nur eine Lösung, $x=0$, also ist $x=0$ das einzige Element mit einer Inversen, diese ist $x^{-1} = 0$, also das Element selbst.


\section{Ungleichungen}
\subsection{a)}
\begin{gather}
	|3-2x| = \sqrt{(3-2x)^2} < 5\\
	\Rightarrow -5 < 3-2x < 5\\
	\Rightarrow 8 > 2x > -2\\
	\Rightarrow 4>x>-1
\end{gather}
Somit muss $x\in M = \{m\in\mathbb{R}| -1 < m < 4\}$

\subsection{b)}

\begin{gather}
	\frac{x+4}{x-2} < x\\
	x+ 4 < x^2 - 2x\\
	-x^2 + 3x + 4 < 0\\
	x^2 - 3x - 4 > 0\\
	(x - 1.5)^2 - 6.25 > 0\\
	(x -1.5)^2 > 6.25\\
	-2.5 > x-1.5 > 2.5\\
	-1 > x > 4
\end{gather}
Das heißt dass $x$ kleiner als $-1$ und größer als $4$ ist. Das heißt dass die ungleichung wahr für alle Nummern außer für die Nummern $-1$, $4$ und alle dazwischen ist.

\subsection{c)}
\begin{gather}
	|(x+4)/(x-2)| < x\\
	\frac{x+4}{x-2} < x,\;\;\;\frac{-x-4}{x-2} < x\\
	x+4 < x^2 - 2x, \;\;\; -x-4 < x^2-2x\\
	0 < x^2 - 3x - 4, \;\;\; x^2 - x + 4 > 0\\
	0 < (x+1)(x-4), \;\;\; x^2 - x + 4 > 0
\end{gather}
Die rechte Seite ist wahr für alle x, also müssen wir nur die linke auswerten. $(x+1)(x-4)$ ist $0$ für $x=4$ und $x=-1$. Für $x=0$ finden wir $1\cdot-4 = -4$, also ist die ungleichung wahr außerhalb von $x=4$ und $x=-1$. Allerdings müssen wir zurück zur ersten ungleichung, und wahrnehmen dass Negative Zahlen nicht in der Lösungsmenge sein können, da der Betrag nicht negativ werden kann, und somit der betrag nicht kleiner als eine negative Zahl sein kann.

Daher muss $x>4$ sein.

\subsection{d)}

\begin{gather}
	x(2-x) > 1 + |x|\\
	-x^2 + 2x - 1 > |x|\\
	x^2 - 2x + 1 < -|x|\\
	(x - 1)^2 < -|x|\\
	(x-1)^2 < x, \;\;\; (x-1)^2 < -x\\
	x^2 - 3x + 1 < 0, \;\;\; x^2 - x + 1 < 0\\
\end{gather}
Da die rechte Seite für kein $x$ wahr ist, können wir hier aufhören, da die Lösung die Kombination aus beiden Seiten ist. Daher ist die Lösungsmenge die leere Menge.


\section{$F_2$}
\subsection{a)}

Die 1 lässt das Element Invariant unter Multiplikation. Daher ist $u$ die 1.\\\\
Die 0 lässt das Element Invariant unter Addition. Daher ist $g$ die 0.

\subsection{b)}

Es gilt:

\begin{equation}
	u + u = u + (-u) = 0
\end{equation}

$u$ ist zu sich selbst invers. Da $u\neq 0$ ist $u>0$ und $-u>0$, wa dem ersten Anordnungsaxiom widerspricht.

Wähle $a = u$ und $b=u$, dann sind
\begin{gather}
	a > u \text{ und } b > u \text{, allerdings}\\
	a + b = u + u = g = 0
\end{gather}
Somit ist auch das zweite Anordnungsaxiom nicht erfüllt und es gibt keine Teilmenge positiver Zahlen, sodass die Anordnungsaxiome erfüllt sind.

\newpage

\newpage

Warum scrollst du noch weiter

\newpage

\newpage

never gonna give you up\\
never gonna let you down
\end{document}
