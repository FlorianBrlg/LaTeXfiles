\documentclass[]{scrartcl}

\usepackage{\string~"/LaTeX/StylePackage"}

\title{Analysis 1 - Hausaufgabe 2}
\author{Florian, Jan}
\date{\today}


\begin{document}

\maketitle
\newpage
\tableofcontents
\newpage

\section{Wachstum der Fakultät}

\section{Rechenregeln für Suprema}

\section{Cauchy-Schwarz Ungleichung}

\textit{Beweis:} Es gilt für alle $\mu,\lambda\in \mathbb{R}$
\begin{equation}
	0 \leq \sum_i (\lambda x_i + \mu y_i)^2
\end{equation}
somit
\begin{gather}
	0 \leq \lambda^2 \underbrace{\sum_i x_i^2}_{=A} + 2\lambda\mu\underbrace{\sum_i x_iy_i}_{=C} + \mu^2 \underbrace{\sum_i y_i^2}_{=B}
\end{gather}
Sei $A$ oder $B$ null, dann sind $x_i$ oder $y_i$ null für alle $i$, somit ist $x_iy_i$ null für alle $i$. Dann ist
\begin{equation}
	C^2 \leq AB
\end{equation}
Sei $AB\neq0$,

Wähle $\lambda = \sqrt B, \mu = \epsilon\sqrt A$ mit $\epsilon\in\{-1,1\}$, $\epsilon C \leq 0$. Dann
\begin{gather}
	0 \leq \lambda^2 A + 2\lambda\mu C + \mu^2 B\\
	0 \leq BA + 2\sqrt A \epsilon\sqrt B C + \underbrace{\epsilon^2}_{=1} AB\\
	0 \leq 2AB + 2\sqrt A\sqrt B \epsilon C\\
\end{gather}
Teile durch $2\sqrt A\sqrt B \neq0$
\begin{gather}
	0 \leq \sqrt A\sqrt B + \epsilon C\\
	-\epsilon C \leq \sqrt A\sqrt B\\
	C^2 \leq AB
\end{gather}
in beiden Fällen gilt die Cauchy Schwartz Ungleichung. Dies endet den Beweis.


\section{$\mathbb{C}$}

\section{Rechnen in $\mathbb{C}$}

\section{Möbiustransformation}
\subsection{a)}
Wir nehmen aus der vorherigen aufgabe die Möbius trafo
\begin{equation}
	\frac{z-a}{1-\bar a z}
\end{equation}

\end{document}

