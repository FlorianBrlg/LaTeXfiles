\subsection{Def: }
Sei $\gamma$ eine kurve. Sei $\phi:[\alpha,\beta]\rightarrow[a,b]$ bijektiv stetig. Dann ist $g=\gamma\cdot\phi$ eine Kurve. Falls $\phi,\phi^{-1}\in C^1$ dann ist dies eine $C^1$ Parameter transformation

\subsection{Def: Orientierungstreu}
$\phi$ heißt orientierungstreu (umkehrend) falls $\phi$ streng monoton wachsend (fallend) ist.

\subsection{Satz: }
Sei $\gamma:[a,b]\rightarrow\mathbb R^n$, $\gamma\in C^1$. Sei $\phi$ eine $C^1$ Parametertransformation.
\begin{gather}
	g(s) = \gamma(\phi(s))\Rightarrow L = \int_a^b |\gamma'(t)|\text dt = \int_\alpha^\beta|g'(s)|\text ds
\end{gather}
\textit{Beweis :} oBdA, $\phi'>0$ 
\begin{gather}
	\int_\alpha^\beta |g'(s)|\text ds = \int_\alpha^\beta |\gamma'(\phi(s))||\phi'(s)|\text ds\\
	=\int_\alpha^\beta |\gamma'(\phi(s))|\phi'(s)\text ds = \int_a^b|\gamma'(t)|\text dt
\end{gather}

\subsubsection{Bemerkung}
Zu jeder regulären Kurve lässt sich eine Umparametrisierung finden, so dass $|g'(s)|=1$. Das heißt, dass die Kurve nach der Bogenlänge parametrisiert ist.

\section{Partielle Ableitung}
$\Omega\subset \mathbb R^n$ ist offen, und $f:\Omega\subset\mathbb R^n \rightarrow \mathbb R^m$

\subsubsection{Beispiele:}
Sei $m=n =1$. $\Gamma_f = \left\{(x,f(x))|x\in\Omega\right\}.$

\begin{itemize}
	\item $\Gamma \subset \mathbb R^{n+1}$ und $N_f(x) =\left\{x\in\Omega|f(x) = c\right\}$
	\item Elektrisches feld einer Raumladung. $q$ in $x_0\in\mathbb R^2$.
		$$
		f(x) = q\frac{(x-x_0)}{|x-x_0|^3}
		$$
	\item reelle darstellung von $e^z$
		\begin{equation}
			f(x,y) = 
			\begin{pmatrix}
				e^x\cos(y)\\
				e^x\sin(y)
			\end{pmatrix}
		\end{equation}
	\item Parametrisierung der Sphäre. $f:(-\pi,\pi)\times(\frac{-\pi}{2},\frac{\pi}{2})\rightarrow S^2$
		$$
		f(\phi,\theta)=
		\begin{pmatrix}
			\cos\phi\cos\theta\\
			\sin\phi\cos\theta\\
			\sin\theta
		\end{pmatrix}
		$$
\end{itemize}

\subsection{Ableitung}
Erste Idee: Halten $n-1$ variablen fest und differenzieren nach der freien Variable. z.B.
$f:\mathbb R^2 \rightarrow \mathbb R, g(x) = f(x,y)\text{ mit y fest}, g:\mathbb R \rightarrow \mathbb R$.
Das ist eine Partielle Ableitung
\subsection{Def: }
ist $x\in\Omega$ partiell differenzierbar nach $x_i$, falls
\begin{equation}
	\partial_i f(x) = \lim_{t\rightarrow 0}\frac{f(x+te_i) -f(x)}{t}
\end{equation}
existiert.

\subsection{Def: }
$f$ heißt stetig partiell differenzierbar, falls $\partial_i f(x)$ für alle $x\in\Omega$ und für alle $i$ existiert und stetig ist

\subsection{Def: Richtungsableitung}
Sei $v\in\mathbb R^n$ ein vektor mit $|v| = 1$ Dann ist
\begin{equation}
	\partial_v f(x) = \lim_{t\rightarrow 0} \frac{f(x+tv) - f(x)}{t}
\end{equation}
die Richtungsableitung von $f(x)$ in Richtung $v$.

\subsection{Def: Gradient von $f$}
sei $f:\Omega\rightarrow\mathbb R$ partiell differenzierbar. Dann ist
\begin{equation}
	\nabla f(x) =
	\begin{pmatrix}
		\partial_1 f(x)\\
		\vdots\\
		\partial_n f(x)
	\end{pmatrix}
\end{equation}
der Gradient von $f$. Wir können schreiben
$$
\partial_v r(x) = \langle \nabla r(x), v\rangle
$$

\subsection{Def: Jacobi Matrix}
$f:\Omega\rightarrow \mathbb R^m$ heißt partiell differenzierbar falls jedes $f_i$ partiell differenzierbar ist.
\begin{equation}
	\partial_i f(x) =
	\begin{pmatrix}
		\partial_i f_1(x)\\
		\vdots\\
		\partial_i f_m(x)
	\end{pmatrix}
\end{equation}
Dies definiert eine Matrix, die Jacobi Matrix

\begin{equation}
	Df(x) =
	\begin{pmatrix}
		\partial_1 f_1(x) & \cdots & \partial_n f_1(x)\\
		\vdots  & & \vdots\\
		\partial_1 f_m(x) & \cdots & \partial_n f_m(x)
	\end{pmatrix}
\end{equation}

falls $m=n$ dann ist $J_f(x) = \det Df(x)$ die Jacobi oder Funktionaldeterminante.








