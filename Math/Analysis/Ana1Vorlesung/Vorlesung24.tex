\documentclass[]{scrartcl}

\usepackage{\string~"/LaTeX/StylePackage"}

\title{Analysis 1 - Vorlesung 24}
\author{}
\date{8.1.2025}


\begin{document}

\maketitle
\newpage
\tableofcontents
\newpage

\subsection{Hauptsatz der Differentialrechnung} 
sei $f\in C^0([a,b])$, dann
\begin{itemize}
	\item[a)] $c\in[a,b]$, $F(x) = \int_c^x f(t)\text dt$ $\Rightarrow$ $F\in C^1([a,b])$
		\item[b)] $F$ sei die stammfunktion von $f$, dann $\int_a^b f(x) \text dx = F(b)-F(a)$
\end{itemize}

\section{Partielle Integration}

\subsection{Satz 9.24}
Seien $f,g\in C^1([a,b])$, dann ist 
$$
\int_a^b f'(x)g(x) = -\int_a^b f(x)g'(x) + (f(b)g(b) - f(a)g(a))
$$
\textit{Beweis: }
$$
f(b)g(b)-f(a)g(a) = \int_a^b (fg)'(x)\text dx = \int_a^b f'g \text dx + \int_a^b fg' \text dx
$$

\subsection{Korollar 9.25}
seien $f\in C^0([a,b]), g\in C^1$. Dann
$$
\int_a^b F(x)g(x) \text dx = -\int f(x)g(x)\text dx + F(x)g(x)|_{x=a}^b
$$

\section{9.6 Substitutionsregel}
\subsection{Satz 9.26}
Seien $I,J$ abgeschlossene beschränkte Intervalle. Sei $f\in C^0(I)$ und $\phi\in C^1(J)$ mit $\phi(J)\subset I$. Dann gilt $\forall \alpha,\beta \in J$
$$
\int_\alpha^\beta f(\phi(t))\phi'(t)\text dt = \int_{\phi(\alpha)}^{\phi(\beta)} f(x)\text dx
$$
\textit{Beweis: }
Sei $F$ die Stammfunktion von $f$. Sei $g(t) = F(\phi(t))$. Dann ist $g'(t) = F'(\phi(t))\phi'(t) = f(\phi(t))\phi'(t).$ Somit ist
$$
\int_\alpha^\beta f(\phi(t))\phi'(x) \text dt = \int_\alpha^\beta g'(x)\text dt = g(\beta) - g(\alpha) = F(\phi(\beta)) - F(\phi(\alpha)) = \int_{\phi(\alpha)}^{\phi(\beta)}f(x)\text dx
$$

\subsubsection{Beispiel}
Verschiebung
$$
\int_a^b f(t+c)\text dt = \int_{a+c)}^{b+c}f(x)\text dx
$$
Multiplikation
$$
\int_a^b f(cx)\text dx = \frac{1}{c}\int_{ca}^{cb} f(x)\text dx
$$

\section{9.7 Uneigentliche Integrale 1 (unbeschränkte Integrale)}
Ziel: Integral über unbeschränkte Integrale definieren, zum Beispiel
$$
\int_0^\infty e^{-x^2}\text dx 
$$
\subsection{Definition Uneigentliches Integral 9.27}
Sei $I = [a,\infty)$ ein Intervall mit $a\in\mathbb R$. Sei $f\in R([a,b]) \;\;\; \forall b<\infty, b>a$
dann ist
$$
\int_0^\infty f(x)\text dx = \lim_{b\rightarrow\infty}\int_a^b f(x)\text dx
$$
falls dieser limes existiert. Dieses Integral heißt dann uneigentliches Integral von $f$ über $[a,\infty)$. Ansonsten ist das Integral divergent.

\subsection{Satz 9.28}
Das Uneigentliche Integral $\int_a^\infty f(x)\text dx$ existiert genau dann, wenn $\forall \epsilon>0 \exists \xi \geq a$ s.d.
$$
|\int_b^{b'} f(x) \text dx| < \epsilon \;\;\;\; \forall b,b' > \xi
$$

\subsection{Definition Absolut Konvergentes Uneigentliches Integral 9.29}

Das Integral $\int_a^\infty f(x)\text dx$ heißt absolut konvergent, falls $\int_a^\infty |f(x)|\text dx$ konvergiert.

\subsection{Satz 9.30}
Falls ein Integral absolut konvergent ist, ist es auch konvergent.

\textit{Beweis: }
$$
|\int_b^{b'} f(x) \text dx| \leq \int_b^{b'}|f(x)|\text dx
$$

\subsection{Satz 9.31 Majorantenkriterium für Integrale}
Sei $g\in R([a,b]),\;\;\;\forall b<\infty,\; g\geq 0,\; \int_a^\infty g(x)\text dx\text{ konvergiert}$
dann existiert ein $y\geq a$ für das $|f(x)|\leq g(x)$ für alle $x$. Dann konvergiert $\int_a^\infty f(x) \text dx$ absolut

\textit{Beweis:} Sei $\epsilon >0$.
\begin{gather*}
	\int_b^{b'}|f(x)|\text dx \leq \int_b^{b'}g(x)\text dx < \epsilon \;\; \forall b,b' \geq \text{max}(\xi,y)
\end{gather*}






\end{document}

