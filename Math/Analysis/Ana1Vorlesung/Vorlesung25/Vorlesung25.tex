\documentclass[]{scrartcl}

\usepackage{\string~"/LaTeX/StylePackage"}

\title{Vorlesung 25 - Analysis 1}
\author{}
\date{13.1.2025}


\begin{document}

\maketitle
\newpage
\tableofcontents
\newpage

 \subsection{Satz 9.34}

$$
	\int_{-\infty}^\infty |f(x)| \text dx
$$
konvergiert, falls
$$
	\int_{-R}^R |f(x)| \text dx \;\;\;\;\forall R<\infty
$$

\section{9.8 Uneigentliche Integrale Teil 2 (Unbeschränkte Funktionen)}

\subsection{}
Sei $f:[a,b)\rightarrow \mathbb R$ beschränkt und Riemann Integrierbar (Auf kompakten Teilintervallen), aber $f(x)$ konvergiert gegen $\pm\infty$ falls $x$ gegen $a$ oder $b$ geht.

\subsection{Definition 9.35}
Falls
$$
\lim_{\xi\rightarrow b} \int_a^\xi f(x)\text dx \left(\lim_{\xi\rightarrow a} \int_\xi^b f(x)\text dx\right)
$$
konvergiert, dann sagen wir dass $\int_a^b f(x)\text dx$ konvergiert, und setzen
$$
\int_a^b f(x) \text dx = \lim_{\xi\rightarrow b}\int_a^\xi f(x)\text dx
$$
\\\\
$f:[a,b)\rightarrow \mathbb R$ beschränkt auf allen kompakten Teilintervallen: $f$ beschränkt ist auf $[a,b-\delta]\forall \delta>0$

\subsection{Satz 9.36}
Falls $|f(x)| \leq \phi(x)\;\;\;\forall x\in[a,b)$ und falls $\int_a^b \phi(x) \text dx$ konvergiert, dann konvergiert $\int_a^b f(x)\text dx$ absolut.

\subsection{Definition 9.37}
Falls für ein $x\in (a,b)$ gilt dass $f(x)\rightarrow\pm\infty$ für $x\rightarrow c$, dann sagen wir dann $\int_a^b f(x) \text dx$ konvergiert, falls $\int_a^c f(x)\text dx$ und $\int_b^c f(x)\text dx$ konvergieren, dann ist
$$
\int_a^b f(x) \text dx = \int_a^c f(x)\text dx + \int_c^b f(x)\text dx
$$

\subsection{Cauchy Hauptwert}
Der Cauchy Hauptwert ist definiert als
$$
\lim_{\epsilon\rightarrow 0}\int_{(a,b|\backslash (c-\epsilon, c+\epsilon)}f(x)\text dx
$$
Der Cauchy Hauptwert kann konvergieren ohne dass $\int_a^c f(x)\text dx$ und $\int_c^b f(x)\text dx$ konvergieren.

\subsubsubsection{Bsp}
$$
\int_{-1}^1 \frac{1}{x}\text dx 
$$

\section{9.9 Riemannsche Integralkriterium}
Sei $f:[1,\infty)\rightarrow[0,\infty)$ monoton falled, und definiere $a_n := f(n)$

\subsection{Satz 9.38}
$\sum a_n$ konvergiert genau dann wenn $\int_1^\infty f(x)\text dx$ konvergiert.

\textit{Beweis: }
$$
\sum_{n=2}^{N+1}a_n \leq \int_1^{N+1}f(x)\text dx \leq \sum_{n=1}^N a_n
$$








\end{document}

